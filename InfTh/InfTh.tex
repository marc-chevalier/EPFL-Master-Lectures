\documentclass[a4paper,10pt]{report}

\usepackage[english]{babel}

\setcounter{secnumdepth}{4}
\setcounter{tocdepth}{4}
\makeatletter
\newcounter {subsubsubsection}[subsubsection]
\renewcommand\thesubsubsubsection{\thesubsubsection .\@arabic\c@subsubsubsection}
\newcommand\subsubsubsection{\@startsection{subsubsubsection}{4}
    {\z@}%
    {-3.25ex\@plus -1ex \@minus -.2ex}%
    {1.5ex \@plus .2ex}%
    {\normalfont\normalsize\bfseries}
}
\renewcommand\paragraph{\@startsection{paragraph}{5}
    {\z@}%
    {3.25ex \@plus1ex \@minus.2ex}%
    {-1em}%
    {\normalfont\normalsize\bfseries}
}
\renewcommand\subparagraph{\@startsection{subparagraph}{6}
    {\parindent}%
    {3.25ex \@plus1ex \@minus .2ex}%
    {-1em}%
    {\normalfont\normalsize\bfseries}
}
\newcommand*\l@subsubsubsection{\@dottedtocline{4}{10.0em}{4.1em}}
\renewcommand*\l@paragraph{\@dottedtocline{5}{10em}{5em}}
\renewcommand*\l@subparagraph{\@dottedtocline{6}{12em}{6em}}
\newcommand*{\subsubsubsectionmark}[1]{}
\makeatother

\usepackage[colorlinks=true]{hyperref} %http://www.ctan.org/tex-archive/macros/latex/contrib/hyperref/
\hypersetup{urlcolor=black,linkcolor=black}

\makeatletter
\def\toclevel@subsubsubsection{4}
\def\toclevel@paragraph{5}
\def\toclevel@subparagraph{6}
\makeatother

\usepackage[utf8]{inputenc}
\usepackage[T1]{fontenc}
\usepackage[table]{xcolor}
\usepackage{mathpazo} %http://www.ctan.org/tex-archive/fonts/mathpazo
\usepackage{stmaryrd} %http://www.ctan.org/pkg/stmaryrd
\usepackage{amsmath} %http://www.ctan.org/pkg/amsmath
\usepackage{amssymb}
\usepackage{mathrsfs}
\usepackage{boiboites}

\usepackage{amsthm} %http://www.ctan.org/pkg/amsthm
\usepackage{proof}
\usepackage{auto-pst-pdf}

\usepackage{footmisc} %http://www.ctan.org/tex-archive/macros/latex/contrib/footmisc

\usepackage{enumerate}
\usepackage{ulem} %http://www.ctan.org/tex-archive/macros/latex/contrib/ulem
\normalem
\usepackage{cancel} %http://www.ctan.org/tex-archive/macros/latex/contrib/cancel

\usepackage{fullpage} %http://www.ctan.org/tex-archive/macros/latex/contrib/preprint/
\setlength{\parindent}{0pt}
\setlength{\parskip}{\medskipamount}

\renewcommand{\thefootnote}{\fnsymbol{footnote}}

\usepackage{pgffor}
\usepackage{tikz}
\usetikzlibrary{arrows,shapes.arrows, chains, positioning, automata, graphs}

\usepackage[ruled,vlined,english]{algorithm2e}
\providecommand{\SetAlgoLined}{\SetLine}
\providecommand{\DontPrintSemicolon}{\dontprintsemicolon}

\usepackage{comment} %http://www.ctan.org/tex-archive/macros/latex/contrib/comment
\usepackage{multirow} %http://www.ctan.org/tex-archive/macros/latex/contrib/multirow

\usepackage{placeins}
\usepackage{wasysym}
\usepackage{qtree}

\newboxedtheorem[boxcolor=red, background=red!5, titlebackground=red!50,titleboxcolor = black]{theorem}{Theorem}{TheC}
\newboxedtheorem[boxcolor=orange, background=orange!5, titlebackground=orange!50,titleboxcolor = black]{definition}{Definition}{DefC}
\newboxedtheorem[boxcolor=blue, background=blue!5, titlebackground=blue!20,titleboxcolor = black]{proposition}{Proposition}{TheC}
\newboxedtheorem[boxcolor=cyan, background=cyan!5, titlebackground=cyan!20,titleboxcolor = black]{corollary}{Corollary}{TheC}
\newboxedtheorem[boxcolor=black, background=black!0, titlebackground=black!20,titleboxcolor = black]{remark}{Remark}{RemC}
\newboxedtheorem[boxcolor=green!70!black, background=green!70!black!5, titlebackground=green!70!black!30,titleboxcolor = black]{notation}{Notation}{NotC}
\newboxedtheorem[boxcolor=yellow, background=yellow!0, titlebackground=yellow!30,titleboxcolor = black]{example}{Example}{ExeC}
\newboxedtheorem[boxcolor=magenta, background=magenta!5, titlebackground=magenta!30,titleboxcolor = black]{lemma}{Lemma}{TheC}
\newboxedtheorem[boxcolor=magenta, background=magenta!5, titlebackground=magenta!30,titleboxcolor = black]{fact}{Fact}{TheC}
\newboxedtheorem[boxcolor=cyan, background=cyan!5, titlebackground=cyan!20,titleboxcolor = black]{consequence}{Consequence}{TheC}
\newboxedtheorem[boxcolor=black, background=black!0, titlebackground=black!20,titleboxcolor = black]{algo}{Algorithm}{AlgoC}
\newboxedtheorem[boxcolor=black, background=black!0, titlebackground=black!20,titleboxcolor = black]{conjecture}{Conjecture}{ConjC}

\newcommand{\EE}{\mathbb{E}}
\newcommand{\CC}{\mathbb{C}}
\newcommand{\GG}{\mathbb{G}}
\newcommand{\KK}{\mathbb{K}}
\newcommand{\NN}{\mathbb{N}}
\newcommand{\PP}{\mathbb{P}}
\newcommand{\QQ}{\mathbb{Q}}
\newcommand{\RR}{\mathbb{R}}
\newcommand{\ZZ}{\mathbb{Z}}

\newcommand{\A}{\mathcal{A}}
\newcommand{\B}{\mathcal{B}}
\newcommand{\C}{\mathcal{C}}
\newcommand{\E}{\mathcal{E}}
\newcommand{\F}{\mathcal{F}}
\newcommand{\G}{\mathcal{G}}
\renewcommand{\H}{\mathcal{H}}
\newcommand{\I}{\mathcal{I}}
\newcommand{\K}{\mathcal{K}}
\renewcommand{\L}{\mathcal{L}}
\newcommand{\M}{\mathcal{M}}
\newcommand{\N}{\mathcal{N}}
\newcommand{\R}{\mathcal{R}}
\renewcommand{\S}{\mathcal{S}}
\newcommand{\T}{\mathcal{T}}
\newcommand{\U}{\mathcal{U}}
\newcommand{\V}{\mathcal{V}}
\newcommand{\W}{\mathcal{W}}
\newcommand{\X}{\mathcal{X}}

\newcommand{\bA}{\mathbf{A}}
\newcommand{\bE}{\mathbf{E}}
\newcommand{\bF}{\mathbf{F}}
\newcommand{\bG}{\mathbf{G}}
\newcommand{\bN}{\mathbf{N}}
\newcommand{\bU}{\mathbf{U}}
\newcommand{\bX}{\mathbf{X}}

\newcommand{\ens}[1]{\left\{ #1 \right\}}
\newcommand{\set}[1]{\left\{ #1 \right\}}
\renewcommand{\leq}{\leqslant}
\renewcommand{\geq}{\geqslant}
\renewcommand{\le}{\leqslant}
\renewcommand{\ge}{\geqslant}
\newcommand{\cplx}[1]{\mathcal O \left( #1 \right)}
\newcommand{\floor}[1]{\left \lfloor #1 \right \rfloor}
\newcommand{\ceil}[1]{\left\lceil #1 \right\rceil}
\newcommand{\brackets}[1]{\left\llbracket #1 \right\rrbracket}
\renewcommand{\angle}[1]{\left\langle #1 \right\rangle}
\newcommand{\donne}{\rightarrow}
\newcommand{\gives}{\rightarrow}
\newcommand{\dans}{\to}
\newcommand{\booleen}{\set{0,1}^*}
\newcommand{\eps}{\varepsilon}
\renewcommand{\implies}{~\Rightarrow~}
\newcommand{\tildarrow}{\rightsquigarrow}
\newcommand{\blank}{\texttt{\char32}}
\newcommand{\trans}[1]{\xrightarrow{#1}}
\newcommand{\rules}[1]{\xrightarrow{#1}}
\newcommand{\todo}[1]{\Large\textcolor{red}{#1}\normalsize}
\newcommand{\wtf}[1]{\Large\textcolor{red}{WTF ?! #1}\normalsize}
\newcommand{\argmin}{\text{argmin}}
\newcommand{\rainbowdash}{\vdash}
\newcommand{\notrainbowdash}{\nvdash}
\newcommand{\rainbowDash}{\vDash}
\newcommand{\notrainbowDash}{\nvDash}
\newcommand{\Rainbowdash}{\Vdash}
\newcommand{\notRainbowdash}{\nVdash}
\newcommand{\bottom}{\bot}
\newcommand{\ra}{\rightarrow}
\newcommand{\Ra}{\Rightarrow}
\newcommand{\longra}{\longrightarrow}
\newcommand{\longRa}{\Longrightarrow}
\newcommand{\la}{\leftarrow}
\newcommand{\La}{\Leftarrow}
\newcommand{\longla}{\longleftarrow}
\newcommand{\longLa}{\Longleftarrow}
\newcommand{\lra}{\leftrightarrow}
\newcommand{\LRa}{\Leftrightarrow}
\newcommand{\longlra}{\longleftrightarrow}
\newcommand{\longLRa}{\Longleftrightarrow}
\newcommand{\opname}[1]{\operatorname{#1}}
\newcommand{\suml}{\sum\limits}
\newcommand{\prodl}{\prod\limits}
\newcommand{\liml}{\lim\limits}
\newcommand{\supl}{\sup\limits}
\newcommand{\infl}{\inf\limits}
\newcommand{\maxl}{\max\limits}
\newcommand{\minl}{\min\limits}
\newcommand{\bigcapl}{\bigcap\limits}
\newcommand{\bigcupl}{\bigcup\limits}
\newcommand{\card}[1]{\left\lvert#1\right\rvert}


%InfTh
\newcommand{\length}[1]{\left\lvert#1\right\rvert}

%Optim
\newcommand{\Det}{\operatorname{Det}}
\newcommand{\TU}{\mathcal{TU}}

%Verif
\newcommand{\ifb}{\mathbf{\ if \ }}
\newcommand{\thenb}{\mathbf{\ then \ }}
\newcommand{\elseb}{\mathbf{\ else \ }}
\newcommand{\dob}{\mathbf{\ do \ }}
\newcommand{\whileb}{\mathbf{\ while \ }}
\newcommand{\abortb}{\mathbf{\ abort \ }}
\newcommand{\skipb}{\mathbf{\ skip \ }}
\newcommand{\inb}{\mathbf{\ in \ }}
\newcommand{\withb}{\mathbf{\ with \ }}
\newcommand{\raiseb}{\mathbf{\ raise \ }}
\newcommand{\trapb}{\mathbf{\ trap \ }}

%CalcForm
\newcommand{\DFT}{\operatorname{DFT}}
\newcommand{\val}{\operatorname{val}}
\newcommand{\Ker}{\operatorname{Ker}}
\newcommand{\pgcd}{\operatorname{pgcd}}
\newcommand{\Tr}{\operatorname{Tr}}

%Compil
\def\changemargin#1#2{\list{}{\rightmargin#2\leftmargin#1}\item[]}
\let\endchangemargin=\endlist

%AlgoPar
\newcommand{\forto}{\text{\bf\ to\ }}


%EvalPerf
\newcommand{\Var}[1]{\text{Var}\left( #1 \right)}
\newcommand{\prob}[1]{\PP\left( #1 \right)}
\newcommand{\esp}[1]{\EE\left[ #1 \right]}
\newcommand{\dd}{\mathrm{d}}


%SystDist
\newcommand{\Receive}{\texttt{Receive~}}
\newcommand{\Send}{\texttt{Send~}}


%Preuves
\newcommand{\betaeq}{=_\beta}
\newcommand{\betared}{\vartriangleright_\beta}
\newcommand{\parabetared}{\vartriangleright_{||\beta}}
\newcommand{\Ackermann}{\A}
\newcommand{\Ter}{\mathcal{T}er}


%Cplx
\newcommand{\Time}{\textsc{Time}}
\newcommand{\TIME}{\textsc{Time}}

\newcommand{\dtime}{\textsc{DTime}}
\newcommand{\dTime}{\textsc{DTime}}
\newcommand{\DTime}{\textsc{DTime}}

\newcommand{\ntime}{\textsc{NTime}}
\newcommand{\nTime}{\textsc{NTime}}
\newcommand{\NTime}{\textsc{NTime}}

\renewcommand{\P}{\textsc{P}}
\newcommand{\coP}{co\text{-}\textsc{P}}

\newcommand{\pTime}{\textsc{PTime}}
\newcommand{\PTime}{\textsc{PTime}}

\newcommand{\NP}{\textsc{NP}}

\newcommand{\npTime}{\textsc{NPTime}}
\newcommand{\NPTime}{\textsc{NPTime}}

\newcommand{\EXP}{\textsc{Exp}}
\newcommand{\expTime}{\textsc{Exp}}
\newcommand{\ExpTime}{\textsc{Exp}}
\newcommand{\EXPTime}{\textsc{Exp}}

\newcommand{\Space}{\textsc{Space}}

\newcommand{\dSpace}{\textsc{DSpace}}
\newcommand{\DSpace}{\textsc{DSpace}}


\newcommand{\nSpace}{\textsc{NSpace}}\newcommand{\NSpace}{\textsc{NSpace}}

\newcommand{\pSpace}{\textsc{PSpace}}
\newcommand{\PSpace}{\textsc{PSpace}}

\newcommand{\npSpace}{\textsc{NPSpace}}
\newcommand{\NpSpace}{\textsc{NPSpace}}
\newcommand{\NPSpace}{\textsc{NPSpace}}

\newcommand{\SpaceTM}{\textsc{SpaceTM}}

\newcommand{\nL}{\textsc{NL}}
\newcommand{\NL}{\textsc{NL}}

\newcommand{\LL}{\textsc{L}}

\newcommand{\coNP}{co\text{-}\textsc{NP}}

\newcommand{\conL}{co\text{-}\textsc{NL}}
\newcommand{\coNL}{co\text{-}\textsc{NL}}

\newcommand{\npc}{\text{\textit{NP-C}}}

\newcommand{\PH}{\textsc{PH}}

\newcommand{\TISP}{\textsc{TISP}}

\newcommand{\Ppoly}{P_{/poly}}

\newcommand{\BPP}{\textsc{BPP}}
\newcommand{\RP}{\textsc{RP}}
\newcommand{\ZPP}{\textsc{ZPP}}
\newcommand{\coBPP}{co\text{-}\textsc{BPP}}
\newcommand{\coRP}{co\text{-}\textsc{RP}}
\newcommand{\coZPP}{co\text{-}\textsc{ZPP}}
\newcommand{\MA}{\textsc{MA}}

\newcommand{\Size}{\textsc{Size}}
\newcommand{\SIZE}{\textsc{Size}}

\newcommand{\dIP}{\mathrm{d}\textsc{IP}}


\title{Information theory \& coding}
\author{Emre \textsc{Telatar}}
\date{2015-2016}

\begin{document}
\maketitle
\tableofcontents

\newpage

Midterm: 27 oct.

Website: \url{http://ipg.epfl.ch/doku.php?id=en:courses:2015-2016:itc}

\chapter{Source Coding}
    \section{Definitions}
        Output of the source is in discrete time and discrete valued.

\begin{definition}
    The output is a sequence of letters from an (finite) alphabet $\U$.
\end{definition}

We want to decompose the encoder into a source encoder and a channel encoder. So, the number of encoder is linear with respect of the the number of sources and channel and not quadratic.

The source encoder outputs a stream of bits and don't care of the channel and the channel encoder translate the input stream of bits without considering the kind of data.

We can use this architecture without loss of generality.

\bigskip

Source coding : Representation of information sources in bits.

\begin{example}
    $\U = \set{a,b,c,d}$
    \[
        \begin{aligned}
            a &\ra 0\\
            b &\ra 0\\
            c &\ra 0\\
            d &\ra 1
        \end{aligned}
        \qquad \text{bad, "singular"}            
    \]
\end{example}

A source code is a function
\[
    \C : \U \ra \set{0;1}^*
\]

\begin{definition}
    A code $\C$ is called singular  if
    \[
        \exists (u,v) \in \U^2, u \neq v \wedge \C(u) = \C(v)
    \]
\end{definition}

\begin{definition}
    A code $\C$ is called non singular (injective) if
    \[
        \forall (u,v) \in \U^2, u \neq v \Ra \C(u) \neq \C(v)
    \]
\end{definition}

\begin{example}
    \[
        \begin{aligned}
            a &\ra 0\\
            b &\ra 00\\
            c &\ra 1\\
            d &\ra 11
        \end{aligned}
        \qquad \text{still bad, "not uniquely decodable"}            
    \]
\end{example}

\begin{definition}
    Given a code $\C : \U \ra \set{0;1}^*$, define for $n\in\NN^*$
    \[
        \C^n : \U^n \ra \set{0;1}^*
    \]
    as
    \[
        \C^n(u_1\ldots u_n) = \C(u_1) \ldots \C(u_n)
    \]
    and
    \[
        \C^* : \U^* \ra \set{0;1}^*
    \]
    as
    \[
        \C^n(u_1\ldots u_k) = \C(u_1) \ldots \C(u_k)
    \]
\end{definition}

\begin{definition}
    A code $\C$ is said to be uniquely decodable if $\C^*$ is injective (non singular).
\end{definition}

We want our code to be uniquely decodable.

\begin{definition}
    A sequence $u_1\ldots u_m$ is a prefix of $v_1 \ldots v_n$ if $n\leqslant m$ and $\forall i\in\llbracket 1,m\rrbracket, u_i = v_i$.
\end{definition}

\begin{example}
    \[
        \begin{aligned}
            a &\ra 0\\
            b &\ra 10\\
            c &\ra 110\\
            d &\ra 111
        \end{aligned}       
    \]
    
    \[
        \underbrace{110}_{c}\underbrace{110}_{c}\underbrace{0}_{a}\underbrace{111}_{d}\underbrace{10}_{b} \underbrace{0}_{a}
    \]
\end{example}

\begin{definition}
    A code $\C$ is said to be prefix free if $\C(u)$ is not a prefix of $\C(v)$ for all $u\neq v$.
\end{definition}

\begin{theorem}
    A prefix free code is uniquely decodable.
\end{theorem}
\begin{proof}
    Suppose that $\C$ is prefix free but not uniquely decodable. So, there is  $u_1\ldots u_m$ and $v_1\ldots, v_n$ such that $u_1\ldots u_m \neq v_1\ldots, v_n$ and $\C(u_1) \ldots \C(u_m) = \C(v_1) \ldots \C(v_n)$.

    Without loss of generality, we can assume $u_1 \neq v_1$.
    
    Suppose $\C(u_1)$ is longer than $\C(v_1)$ $\Ra$ $\C(v_1)$ is a prefix of $\C(u_1)$: contradiction.
    
    In the other case, $\C(u_1)$ is a prefix of $\C(v_1)$: contradiction.
\end{proof}

\begin{example}
    \[
        \begin{aligned}
            a &\ra 0\\
            b &\ra 01\\
            c &\ra 011\\
            d &\ra 111
        \end{aligned}
        \qquad \text{not prefix free but still uniquely decodable}
    \]
    
    \[
        \underbrace{110}_{c}\underbrace{110}_{c}\underbrace{111}_{d}
    \]
\end{example}

Binary tree representation of source codes.

\begin{center}
	\Tree [.$\emptyset$ [.0 [.00 [000 ] [001 ] ] [.01 [010 ] [011 ] ] ] [.1 [.10 [100 ] [101 ] ] [.11 [110 ] [111 ] ] ] ]
\end{center}

\begin{center}
	\Tree [ [.$\C(b)$ [$\C(a)$ ] [$\C(c)$ ] ] [[ ] [[ ][ ] ] ] ]
	$\ra$ not prefix free.
\end{center}

\begin{center}
    \Tree [[[ ][$\C(a)$ ]][[$\C(b)$ ][[$\C(c)$ ][$\C(d)$ ] ] ] ]
    $\ra$ prefix free
\end{center}

\begin{remark}
    In the binary tree representation of a prefix tree code, all codewords are found on the leaves.
\end{remark}
    \section{\textsc{Kraft}'s inequality}
        \begin{theorem}[\textsc{Kraft}'s inequality for prefix-free codes]
    If $\C$ is prefix free, then
    \[
        \operatorname{KarftSum}(\C) := \sum\limits_{u\in\U} 2^{-\lvert \C(u) \rvert} \leqslant 1
    \]
\end{theorem}
\begin{proof}
    \[
        \begin{aligned}
            1 &= \sum\limits_{l \in \text{leaves}} \prob{\text{squirrel reaches leaf } l}\\
            &= \sum\limits_{l \in \text{leaves}} 2^{-\operatorname{height}(l)}\\
            &\geqslant \sum\limits_{u\in\U} 2^{-\lvert \C(u)\rvert}
        \end{aligned}
    \]
\end{proof}

\begin{proposition}
    \[
        \opname{KraftSum}(\C^n) = (\opname{KraftSum}(\C))^n
    \]
\end{proposition}
\begin{proof}
    \[
        \begin{aligned}
            \opname{KraftSum}(\C^n) &= \sum\limits_{u_1\ldots u_n\in\U^n} 2^{-\underbrace{\left\lvert\C^n(u_1\ldots u_n)\right\rvert}_{\C(u_1)\ldots\C(u_n)}}\\
            &= \sum\limits_{u_1\ldots u_n\in\U^n} 2^{-\left(\sum\limits_{i=1}^n\left\lvert \C(u_i) \right\rvert\right)}\\
            &= \sum\limits_{u_1\ldots u_n\in\U^n} \prod\limits_{i=1}^n 2^{-\left\lvert\C(u_i) \right\rvert}\\
            &= \prod\limits_{i=1}^n \sum\limits_{u_i}2^{-\left\lvert \C(u_i)\right\rvert}\\
            &= (\opname{KraftSum(\C)})^n
        \end{aligned}
    \]
\end{proof}

\begin{example}
	\begin{center}
	    \begin{tabular}{c|c|c}
	        $u$ & $l(u)$ & $\C(u)$\\\hline
	        $a$ & $1$ & \\
	        $b$ & $2$ & \\
	        $c$ & $3$ & \\
	        $d$ & $3$ & \\
	    \end{tabular}
	\end{center}
	
    Is there a prefix free code $\C$ with $\left\lvert \C(u) \right\rvert = l(u)$.
\end{example}

\begin{theorem}["Reverse" \textsc{Kraft} inequality]
    Given an alphabet $\U$ and a function $l:\U \to \NN$ such that $\sum\limits_{u\in\U} 2^{-l(u)} \leqslant 1$ then there exists a prefix free code $\C : \U \to \set{0,1}^*$ such that $\left\lvert \C(u)\right\rvert = l(u)$ for each $u\in\U$.
\end{theorem}
\begin{proof}
    Suppose the alphabet $\U$ has $k$ letters, assume $\U = \llbracket 1;k \rrbracket$. Suppose also without loss of generality thaht $\forall i \in \llbracket 1; k-1 \rrbracket, l(i) \leqslant l(i+1)$.
    
    Consider the following algorithm:

    \begin{enumerate}
        \item Start with the binary tree of height $l(k)$ with all nodes unoccupied.
        \item For $i = 1,\ldots,k$: place $\C(i)$ at an unoccupied node at height $l(i)$ and mark as occupied all nodes that descend from it.
        \item Return $\C$.
    \end{enumerate}
    
    When we try to find a free node in the tree, how many unoccupied nodes are there at height $l(i)$?
    
    \[
        \begin{aligned}
            &= \underbrace{2^{l(i)}}_{\text{we start with}} - \underbrace{2^{l(i)-l(1)}}_{\substack{\text{nodes eliminated}\\ \text{when placing }\C(1)}} - \cdots - \underbrace{2^{l(i)-l(i-1)}}_{\substack{\text{nodes eliminated}\\ \text{when placing }\C(i-1)}}\\
            &= 2^{l(i)} \left( 1-\underbrace{\left(\sum\limits_{k=1}^{i-1} 2^{-l(k)} \right)}_{< 1} \right)\\
            &> 0
        \end{aligned}
    \]
\end{proof}

\begin{proposition}[\textsc{Kraft}'s inequality for non-singular codes]
    Suppose $\C : \U \to \set{0,1}^*$ is a non singular code then
    \[
        \begin{aligned}
            \opname{KraftSum}(\C) &= \sum\limits_{u\in\U} 2^{-\left\lvert \C(u) \right\rvert}\\
            &\leqslant 1+ \max\limits_n \left\lvert \C(u)\right\rvert
        \end{aligned}
    \]
\end{proposition}
\begin{proof}
    Let $l_{max}$ the height of the tree.
    
	    \Tree [  [  [  [ $\vdots$ ] [ $\vdots$ ] ] [  [ $\vdots$ ] [ $\vdots$ ] ] ] [  [  [ $\vdots$ ] [ $\vdots$ ] ] [  [ $\vdots$ ] [ $\vdots$ ] ] ] ]

    Height 0: $2^{-0} = 1$
    
    Height 1: $2^1 2^{-1} = 1$
    
    Height 2: $2^2 2^{-2} = 1$
\end{proof}

\begin{theorem}
    If $\C$ is a uniquely decodable code then $\opname{KraftSum}(\C) \leqslant 1$.
\end{theorem}
\begin{proof}
    Suppose $\C$ is uniquely decodable, in particular $\C^n$ is non singular.
    \[
        \begin{aligned}
            \left (\opname{KraftSum}(\C)\right )^n &= \opname{KraftSum}(\C^n) \leqslant 1+\max\limits_{u_1\ldots u_n} \left \lvert \C(u_1)\ldots\C(u_n) \right \rvert\\
            &= 1 + n \max\limits_u \left \lvert \C(u) \right \rvert\\
            &= \cplx{n}
        \end{aligned}
    \]
    
    If $\opname{KraftSum}(\C)>1$ would be $\cplx{\exp(n)} \Ra \opname{KraftSum}(\C)\leqslant 1$.
\end{proof}
\begin{corollary}
    If $\C$ is a uniquely decodable code, then there exists a prefix free code $\C'$ such that
    \[
        \forall u \in \U, \left\lvert \C'(u) \right\rvert = \left\lvert \C(u) \right\rvert    
    \]
\end{corollary}

    \section{Entropy}
        \begin{definition}
    Expected codeword length
    \[
        \sum\limits_i \prob{a_i} \left\lvert \C(a_i) \right\rvert
    \]
\end{definition}


\begin{definition}[Entropy]
    For a discrete random variable $X$ which has a mass function $P$
    \[
        H(X) := -\sum\limits_i P{x_i}\log(P(x_i))
    \]
\end{definition}

\begin{theorem}
   Expected codeword length $\geqslant$ Entropy
\end{theorem}

\begin{example}
    $\U = \set{a,b,c,d}$.
    \[
        \begin{array}{cccc}
            \prob{a} = \frac{1}{2} &
            \prob{b} = \frac{1}{4} &
            \prob{c} = \frac{1}{8} &
            \prob{d} = \frac{1}{8}
        \end{array}
    \]
    
    \[
        \begin{aligned}
            a &\ra 0\\
            b &\ra 10\\
            c &\ra 110\\
            d &\ra 111
        \end{aligned}
    \]
    
    \[
        \esp{\left\lvert \C(\U) \right\rvert} = 1.75 = H(\U)
    \]
\end{example}

\begin{theorem}
    Given a source $\U$, there exists a prefix code $\C$ such that
    
    \[
        \esp{\left\lvert \C(\U) \right\rvert}\leqslant H(\U) + 1
    \]
\end{theorem}
\begin{proof}
    Take $l(u) = \ceil{\log_2\frac{1}{p(u)}}$

    \[
        \log \frac{1}{p(u)} \leqslant l(u) \leqslant 1 + \log \frac{1}{p(u)}
    \]
    
    Using the left inequality.
    
    \[
        \begin{aligned}
            2^{-l(u)} \leqslant p(u) &\Ra \sum\limits_u 2^{-l(u)} \leqslant \sum\limits p(u) = 1\\
            &\Ra \exists \text{a prefix free }\C:\left\lvert \C(u)\right\rvert = l(u)
        \end{aligned}            
    \]    
    
    
    Also
    \[
        \begin{aligned}
            \esp{\left\lvert \C(\U) \right\rvert} &= \sum\limits_u p(u)l(u)\\
            &\leqslant \sum\limits_u p(u) \left( 1+\log\frac{1}{p(u)}\right)\\
            &=1+H(\U)
        \end{aligned}
    \]
\end{proof}


\begin{proposition}
    \[
        H(\U) \geqslant 0    
    \]
\end{proposition}
\begin{proof}
    \[
        \begin{aligned}
            H(U) &= \sum \underbrace{p(u)}_{\geqslant 0} \overbrace{\log \underbrace{\frac{1}{p(u)}}_{\geqslant 1}}^{\geqslant 0}\\
            &\geqslant 0
        \end{aligned}            
    \]
\end{proof}

\begin{proposition}
    \[
        H(\U) \leqslant \log \left\lvert \U \right\rvert
    \]
\end{proposition}
\begin{proof}
    \[
        \begin{aligned}
            \sum p(u) \log \frac{1}{p(u)} - \sum p(u) \log\left\lvert \U \right\rvert &= \sum p(u) \log \frac{1}{p(u) \left\lvert\U\right\rvert}\\
            &= \left( \sum p(u) \log \frac{1}{p(u)\left\lvert \U \right\rvert} \right) \left( \log e\right)\\
            &\leqslant \left( \log e\right)\left( \sum p(u)\left( \frac{1}{p(u)\left\lvert \U\right\rvert}-1\right) \right)\\
            &=\left( \log e\right)\left( 1-1 \right)\\
            &= 0
        \end{aligned}
    \]
\end{proof}

\begin{proposition}
    Suppose $U$ and $V$ are independent random variables. Then 
    \[
        H(UV) = H(U) + H(V)
    \]
\end{proposition}
\begin{proof}
    We can consider $(U,V)$ as a new random variable.
    
    \[
        \prob{(U,V) = (u,v)} = \frac{\prob{U = u} \prob{V = v}}{p(u,v)}
    \]
    
    Then 
    \[
        \begin{aligned}
            H(UV) &= \sum p(uv) \log \frac{1}{p(uv)}\\
            &= \sum p(uv) \log \frac{1}{p(u)p(v)}\\
            &= \sum p(uv) \left( \log \frac{1}{p(u)} + \log \frac{1}{p(v)}\right)\\
            &= \sum\limits_{u,v} p(uv) \log \frac{1}{p(u)} + \sum\limits_{u,v} p(uv) \log \frac{1}{p(v)}\\
            &= \sum\limits_{u} p(u) \log \frac{1}{p(u)} + \sum\limits_{v} p(v) \log \frac{1}{p(v)}\\
        \end{aligned}
    \]
\end{proof}

Suppose we have a memoryless, stationary source, producing $\U_1$, $\U_2$, $U_3$, $\U_4$...

So fare, we choose to represent the source output in a letter to letter fashion. If instead, we use a code to represent $n$ letters at a time, we will have
\[
    H(\U_1\ldots \U_n) \leqslant \frac{\esp{\left\lvert \C(\U_1\ldots \U_n)\right\rvert}}{n} \leqslant \frac{H(\U_1\ldots \U_n) +1}{n}
\]

Also
\[
    \begin{aligned}
        \frac{1}{n}H(\U_1\ldots \U_n) &= \frac{1}{n} \sum\limits_{i=1}^n H(\U_i)\\
        &= \frac{1}{n}\sum\limits_{i=1}^n H(\U_1)\\
        &= H(\U_1)
    \end{aligned}
\]

So far we have seen bounds to the performance of code design, but we have not seen how to actually design a prefix code.

Given $p:\U \to \RR$. 

$\min\limits_{l:\U\to\NN} \sum p(u) l(u)$ 

$\sum\limits_u 2^{-l(u)} \leqslant 1$

Properties of optimal (in terms of minimizing $\sum p(u) \lvert\C(u)\rvert$ codes

\begin{enumerate}
    \item if $p(u) < p(v)$ then $l(u) \geqslant l(v)$.
        \begin{proof}
            Suppose not $p(u) < p(v)$ but $l(u) < l(v)$. Then swap the codewords for $u$ \& $v$. That improve the code.
        \end{proof}
        
    \item In an optimal prefix code, there are at least 2 longest codewords.
        \begin{proof}
            If not, the longest codeword can be shortened without violating the prefix free condition.
        \end{proof}
        
    \item Among optimal codes, there is one for which the two least probable symbols are siblings.
        \begin{proof}
            They have to be at the same height. We just need to permute in order to make the two least probable symbols siblings.
        \end{proof}
        
    \item 1-to-1 correspondence between prefix free codes for an alphabet $\U$ and guessing strategies for $\U$.
        \[
            \esp{\#\text{ of question to guess}} = \esp{\lvert \text{codeword} \rvert}
        \]
\end{enumerate}

Suppose $U$, $V$ are random variables. Suppose we know $V$ ($V = v$). How much entropy is in $\U$?

We know that under the conditioning $V=v$, the probability that $U=u$ is now $\prob{U=u=\vert V=v} = \frac{p(uv)}{p(v)} =: p(u\vert v)$. So we define

\[
    \begin{aligned}
        H(U\vert V=v) &= \sum_u p(u\vert v) \log \frac{1}{p(u\vert v)}\\
        H(U\vert V) &= \sum_v p(v) H(U\vert V=v)\\
    \end{aligned}
\]

\begin{conjecture}
    \[
        H(U\vert V) \leqslant H(U)
    \]
\end{conjecture}

\begin{conjecture}
    \[
        H(UV) = H(U) + H(V \vert U) = H(V) + H(U \vert V)
    \]
\end{conjecture}

\begin{example}
    $U$, $V$ are binary random variables. $\U = \set{a,b}$, $\V = \set{0,1}$
    
    \begin{tabular}{c|cc}
        $p(uv)$ & $0$ & $1$\\
        \hline
        a & $\frac{1}{2}$ & $\frac{1}{4}$\\
        b & $0$ & $\frac{1}{4}$
    \end{tabular}
    
    \[
        \begin{aligned}
            H(UV) &= \frac{1}{2} \log 2 + \frac{1}{4} \log 4 + \frac{1}{4} \log 4\\
            &= 1.5
        \end{aligned}
    \]

    \[
        \begin{array}{lll}
            H(V) = 1 & H(U\vert V=0) = 0 & H(V\vert U=b) = 0 \\
            H(U) = \frac{3}{4} \log\frac{4}{3} + \frac{1}{4}\log 4 & H(U\vert V=1) = 1 & H(V\vert U=a) = 1 \\
            & H(U\vert V) = \frac{1}{2} & H(V\vert U) = \frac{3}{4}\left( \frac{3}{4} \log\frac{4}{3} + \frac{1}{4}\log 4 \right)
        \end{array}        
    \]
\end{example}

\begin{proof}
    \[
        \begin{aligned}
           p(uv) &= p(u) p(v\vert u)\\
           \Ra \log \frac{1}{p(uv)} &= \log \frac{1}{p(u)} + \log \frac{1}{p(u\vert v)}\\
           H(UV) &= \esp{\log \frac{1}{p(UV)}}\\
           &= \underbrace{\esp{\log\frac{1}{p(U)}}}_{H(U)} + \underbrace{\esp{\log\frac{1}{p(V\vert U)}}}_{H(V\vert U)}
        \end{aligned}            
    \]

    \[
        \begin{aligned}
            H(V\vert U) &= \sum_u p(u) H(V\vert U=u)\\
            &= \sum_u p(u) \sum_v p(v\vert u) \log \frac{1}{p(v\vert u)}\\
            &= \sum_{v,u} p(vu) \log \frac{1}{p(v\vert u)}
        \end{aligned}
    \]
\end{proof}
    \section{Mutual information}
        \begin{definition}
\[
    \begin{aligned}
        H(U) + H(V) - H(UV) &= H(V) - H(V\vert U)\\
        &= H(U) - H(U\vert V)
    \end{aligned}
\]

This difference is the mutual information $I(U,V)$ between $U$ and $V$, and represent the reduction of the effort to guess $U$ by knowing $V$ (and vice-versa by symmetry).
\end{definition}

\begin{lemma}
    Suppose $\W$ an alphabet and $p$ and $q$ are two probability distribution on $\W$, then
    \[
        \sum\limits_w p(w) \log \frac{p(w)}{q(w)}\geqslant 0
    \]
\end{lemma}
\begin{proof}
    We need to prove $\sum p(w) \log \frac{q(w)}{p(w)} \leqslant 0$.
    
    \[
        \begin{aligned}
            \sum p(w) \log \frac{q(w)}{p(w)} &\leqslant \sum p(w) (\frac{q(w)}{p(w)}-1)\\
            &= \sum q(w) - p(w)\\
            &= 1-1\\
            &= 0
        \end{aligned}            
    \]
\end{proof}

\begin{theorem}
    \[
        I(U,V) \geqslant 0    
    \]
\end{theorem}
\begin{proof}
    \[
        \begin{aligned}
            I(U,V) &= H(U) + H(V) - H(U,V)\\
            &= \esp{\log\frac{p(UV)}{p(U)p(V)}}\\
            &= \sum\limits_{u,v} p(uv) \frac{p(uv)}{p(u)p(v)}\\
          &\geqslant 0  
        \end{aligned}            
    \]
\end{proof}

\begin{theorem}
    \[
        H(U_1\ldots U_n) = H(U_1) + H(U_2\vert U_1) + H(U_3\vert U_1 U_2) + \cdots + H(U_n \vert U_1\ldots U_{n-1})
    \]
\end{theorem}
\begin{proof}
    \[
        \begin{aligned}
            p(u_1\ldots u_n) &= p(u_1) p(u_2 \vert u_1) p(u_3\vert u_1u_2) \cdots p(u_n \vert u_1\ldots u_1)\\
            \Ra \log\frac{1}{p(u_1\ldots u_n)} &= \log\frac{1}{p(u_1)} +  \log\frac{1}{p(u_2 \vert u_1)}+ \log\frac{1}{p(u_3\vert u_1u_2)} + \cdots + \log\frac{1}{p(u_n \vert u_1\ldots u_1)}
        \end{aligned}
    \]
\end{proof}

\begin{definition}[Condition mutual information]
    \[
        \begin{aligned}
            I(U,V \vert W) &= H(U\vert W) + H(V\vert W) - H(UV \vert W)\\
             &= H(U \vert W) - H(U \vert V W)\\
             &= H(V \vert W) - H(V \vert U W)
        \end{aligned}
    \]
\end{definition}

\begin{theorem}
    \[
        I(U,V \vert W) \geqslant 0
    \]
    with $=$ when $U$ and $V$ are independent conditional in $W$.
\end{theorem}

\begin{theorem}
    \[
        I(U_1\ldots U_n,V) = I(U_1,V) + I(U_2,V\vert U_1    ) + I(U_3,V\vert U_1, U_2) + \cdots + I(U_n, V \vert U_1\ldots U_n)
    \]
\end{theorem}
\begin{proof}
    \[
        I(U_1\ldots U_n, V) = H(U_1\ldots U_n) - H(U_1\ldots U_n \vert V)    
    \]
    
    \[
        \begin{aligned}
            H(U_1\ldots U_n) &= H(U_1) + H(U_2\vert U_1) + \cdots + H(U_n \vert U_1 \ldots U_{n-1})\\
            H(U_1\ldots U_n \vert V) &= H(U_1 \vert V) + H(U_2\vert U_1 V) + \cdots + H(U_n \vert U_1 \ldots U_{n-1} V)\\
            I(U_1\ldots U_n, V) &= I(U_1,V) + I(U_2 V \vert U_1) + \cdots + I(U_n V \vert U_1 \ldots U_{n-1})
        \end{aligned}
    \]
\end{proof}

\begin{remark}[\textsc{Markov} chains]
    Suppose $X$, $Y$ and $Z$ are random variables.
    
    It is always true that
    \[
        p(x,y,z) = p(x)p(y\vert x) p(z\vert x,y)
    \]
    
    If $X \text{----} Y \text{----} Z$, then
    \[
        p(x,y,z) = p(x)p(y\vert x) p(z\vert y)
    \]
\end{remark}

\begin{proposition}
    If
    \[
        U_1 \text{----} \cdots \text{----} U_n    
    \]
    then
    \[
    H(U_1, \ldots, U_n) = H(U_1) + H(U_2 \vert U_1) + \cdots + H(U_n \vert U_{n-1})
    \]    
\end{proposition}

\begin{example}
    $X$, $Y$, $Z$ are binary random variables
    \begin{center}
        \begin{tabular}{ccc||c}
            $x$ & $y$ & $z$ & $p(x,y,z)$\\
            \hline
            0 & 0 & 0 & $\frac{1}{4}$\\
            0 & 0 & 1 & 0\\
            0 & 1 & 0 & 0\\
            0 & 1 & 1 & $\frac{1}{4}$\\
            1 & 0 & 0 & 0\\
            1 & 0 & 1 & $\frac{1}{4}$\\
            1 & 1 & 0 & $\frac{1}{4}$\\
            1 & 1 & 1 & 0\\
        \end{tabular}
    \end{center}
    
    \[
        \begin{aligned}
            H(X) &= H(Y) = H(Z) = 1\\
            I(X\vert Y) &= H(X) + H(Y) - H(X,Y) = 1 + 1 - 2 = 0\\
            I(X,Y\vert Z) &= H(X\vert Z) - H(X \vert YZ) = 1 - 0 = 1
        \end{aligned}
    \]
\end{example}

\begin{theorem}[Data Processing]
    Suppose $U \text{----} V \text{----} W$, then
    \[
        I(U,W) \leqslant I(U,V)    
    \]
\end{theorem}
\begin{proof}
    \[
        \begin{aligned}
            I(U, VW) &= I(U,V) + I(U,W\vert V)\\
            &= I(U,W) + I(U,V \vert W)\\
            &\geqslant I(U,W)
        \end{aligned}            
    \]
\end{proof}

\begin{corollary}
    If $U \text{----} V \text{----} W \text{----} X$, then
    \[
        I(U,X) \leqslant I(V,W)    
    \]
\end{corollary}
    \section{Entropy rate}
        We observed that the quantity $\frac{1}{n}H(U_1,\ldots,U_n)$ as the number of bits/letters to represent a source that produce $U_1U_2\ldots$, when we encoded $n$ letters at a time, so it makes sense to study $\lim\limits_{n\to\infty} \frac{1}{n} H(U_1\ldots U_n)$.

\begin{definition}
    Given a stochastic process $U_1U_2U_3\ldots$, we define its entropy rate $\H(U) = \lim\limits_{n\to\infty} H(U_1\ldots U_n)$ if the limit exists.
\end{definition}

Recall that when $U_1,\ldots,U_n$ are iid, then
\[
    \begin{aligned}
        H(U_1\ldots U_n) &= H(U_1)+ \cdots + H(U_n\vert U_1\ldots U_{n-1})\\
        &= \sum\limits_{i=1}^n H(U_i)\\
        &= \sum\limits_{i=1}^n H(U_1)\\
        &= nH(U_1)\\
        \Ra \lim\limits_{n\to\infty} \frac{1}{n} H(U_1\ldots U_n) &= H(U_1)
    \end{aligned}
\]

So the entropy rate of an iid stochastic process is $H(U_1)$.

\begin{definition}
    A process $U_1,U_2,\ldots$ is said to be stationary if
    \[
        \forall (k,n)\in\NN^2, (U_1\ldots U_n) \sim (U_{k+1}\ldots U_{k+n})
    \]
\end{definition}

\begin{theorem}
    If $U_1U_2\ldots$ is a stationary process, then the entropy rate exists, and
    \[
        \lim\limits_{n\to\infty} \frac{1}{n} H(U_1\ldots U_n) = \lim\limits_{n\to\infty} H(U_n \vert U_1\ldots U_{n-1})
    \]
\end{theorem}
\begin{proof}
    Let $a_n := H(U_n \vert U_1 \ldots U_{n-1})$ and $b_n := \frac{1}{n} H(U_1 \ldots U_n)$.
    
    We are trying to show
    \begin{enumerate}
        \item $a = \lim\limits_{n\to\infty} a_n$ exists
        \item $\lim\limits_{n\to\infty} b_n = a$
    \end{enumerate}
    
    Let us start with (1). Observe $a_n \geqslant 0$, also
    \[
        \begin{aligned}
            a_{n-1} &= H(U_{n-1} \vert U_1 \ldots U_{n-2})\\
            &= H(U_n \vert U_2 \ldots U_{n-1})\\
            &\geqslant H(U_n \vert U_1 U_2\ldots U_{n-1})\\
            &= a_n
        \end{aligned}            
    \]
    
    So $a_n$ is a non increasing sequence which is low bounded $\Ra$ $\lim_{n\to\infty} a_n$, call
    
    To show (2), observe
    \[
        \begin{aligned}
            b_n &= \frac{1}{n} H(U_1\ldots U_n)\\
            &= \frac{1}{n} \left( H(U_1)+H(U_2 \vert U_1) + \cdots + H(U_n \vert U_1\ldots U_{n-2}) \right)\\
            &= \frac{1}{n} \sum\limits_{i=1}^n a_i
        \end{aligned}
    \]
    By the \textsc{Cesàro}'s theorem, $b_n\to a$.
\end{proof}

\begin{theorem}
    If $U_1U_2\ldots$ is a stochastic process with entropy rate $H$, then for any $\varepsilon > 0$, there exists a source code $\C^n : \U^n \to \set{0,1}^*$ such that the code uses $ \leqslant H + \varepsilon$ bits/source letter on the average
\end{theorem}

We know that for every $n$, there is a prefix free code
\[
    \C_n: \U^n \to \set{0,1}^*
\]
with the property that
\[
    \esp{\length{\C_c(U_1\ldots U_n}} \leqslant \frac{H(U_1\ldots U_n)+1}{n} \underset{n \to \infty}{\to   } H
\]

Consequently there exists some $n$ and a prefix code $\C_n$ such that
\[
    \begin{aligned}
        \frac{1}{n}\esp{\length{\C_n(U_1\ldots U_n)}} &\leqslant H+\varepsilon\\
        \frac{1}{n}\esp{\length{\C_n(U_{n+1}\ldots U_{2n})}} &\leqslant H+\varepsilon\\
        \frac{1}{n}\esp{\length{\C_n(U_{2n+1}\ldots U_{3n})}} &\leqslant H+\varepsilon
    \end{aligned}
\]
    \section{Fixed-to-fixed codes}
        So far we have seen codes of the type
\[
    \U^n \to \set{0,1}^*
\]

They are called Fixed-to-variable-length codes and all have error free recovery of the source from its representation.

We now discuss "Fixed-to-fixed" codes, we want
\[
    \C: \U^n \to \set{0,1}^k
\]

\begin{example}
    Suppose $\U=\set{a,b,c,d}$.
    \[
        \C: \U^n \to \set{0,1}^k
    \]
    $\card{\U^n} = 4^n$ and 
    $\card{\set{0,1}^k} = 2^k$
    
    error free recover requires $k\geqslant 2n$.
\end{example}

To obtain efficient codes, we will give up error-free recovery replace this by recovery with very small probability of error. For this, we will have the code assign binary representation only to a subset $\S \subseteq \U^n$. We will choose this subset to ensure 
\[
    \prob{(U_1\ldots U_n)\in\S} \simeq 1 \wedge \card{\S} \leqslant 2^k
\]

\begin{example}
    Suppose $\U = \set{a,b}p$, $p_a = 0.25$, $p_b = 0.75$. Source is memoryless stationary (iid) $U_1, U_2, \ldots$
    
    What kind of sequences $(u_1,\ldots,u_n)$ do we expect the source to produce?
    
    We expect that $(u_1,\ldots,u_n)$ to have $\simeq \frac{1}{4} n$ a's and $\simeq \frac{3}{4} n$ b's.
    
    Pick $\varepsilon > 0$ small. Let us define $\S = \set{\left.(u_1,\ldots,u_n) \middle\vert \length{(u_1,\ldots,u_n)}_a = \frac{1}{4}n(1\pm \varepsilon)\right.}$ .
    
    Intuitively, we expect $\set{(u_1,\ldots,u_n) \in \S}$ to be a high probability event.
    
    How about the size of $\S$?
    
    How many sequences $(u_1,\ldots,u_n)$ are there with $i$ a's and $n-i$ b's? 
    \[
        \binom{n}{i}
    \]
    So
    \[
        \begin{aligned}
            \card{\S} &= \sum\limits_{i = \ceil{\frac{n}{4}(1-\varepsilon)}}^{\floor{\frac{n}{4}(1+\varepsilon)}} \binom{n}{i}\\
            &\leqslant (n+1)\binom{n}{\floor{\frac{n}{4}(1+\varepsilon)}}
        \end{aligned}
    \]
    \[
        \begin{aligned}
            1 &= (p + (1-p))^n\\
            &=\sum\limits_{i=0}^n \binom{n}{i} p^i (1-p)^{n-i}\\
            &\geqslant \binom{n}{i} p^i (1-p)^{n-i}\\
            \Ra \binom{n}{i} &\leqslant \left(\frac{1}{p}\right)^i\left(\frac{1}{1-p}\right)^{n-i}\\
            &\leqslant \left(\frac{n}{i}\right)^i\left(\frac{n}{n-i}\right)^{n-i} \qquad \text{ with } p=\frac{i}{n}
        \end{aligned}
    \]
    
    \[
        \log\binom{n}{i} \leqslant n\left( \alpha \log\frac{1}{\alpha} + (1-\alpha)\log\frac{1}{1-\alpha} \right)
    \]
    \[
        \binom{n}{\floor{\frac{n}{4}(1+\varepsilon)}} \leqslant 2^{nH(U')}
    \]
    where $U' = \begin{cases}
        a & \text{ wp } \frac{1}{4}(1+\varepsilon)\\
        b & \text{ wp } 1 - \frac{1}{4}(1+\varepsilon)
    \end{cases}$.
    
    So $\card{\S} \leqslant (n+1)2^{nH(U')}$ so if we choose $k = aH(U') + \log(n+1)$ we will have $\card{} \leqslant 2^k$ with $\frac{k}{n} = \underbrace{H(U') + \frac{\log(n+1)}{n}}_{\simeq H(u) \text{if }\varepsilon\text{ is small and }n\text{ is large}}$ we will have "almost free" representation.
\end{example}

\subsection{General case}

We are given an alphabet $\U$ and a distribution $p$ on $\U$, $\set{p(u) \vert u\in\U}$, and we have a source that produces iid letters $U_1,U_2,\ldots$ each with distribution $p$. We want to have a set $T_{n,\varepsilon, p} \subseteq \U^n$ with the properties
\begin{enumerate}
    \item $\prob{(U_1\ldots U_n) \in T_{n,\varepsilon,p}} \simeq 1$
    \item $\card{T_{n,\varepsilon,p}} \leqslant ...$
\end{enumerate}

We say a sequence $(u_1\ldots u_n) \in \U^n$ to be $\varepsilon$-typical with respect to the distribution $p$ if for every $u\in\U, \frac{\card{\set{i \vert u_i = u}}}{n} = p(u) (1\pm \varepsilon)$.

Set $T_{n,\varepsilon, p}$ to be set of $(u_1,\ldots u_n)$'s which are $\varepsilon$-typical wrt $p$.

\begin{proposition}
    \[
        \prob{(U_1\ldots U_n) \not\in T_{n,\varepsilon,p}} \leqslant \frac{1}{n \varepsilon^2p_{\min}}
    \]
\end{proposition}
\begin{proof}
    Let $B_u$ be the vent that $\frac{1}{n}\length((U_1\ldots U_n))_u \not\in [p(u)(1-\varepsilon); p(u)(1+\varepsilon)]$. Then 
    \[
        \begin{aligned}
            \prob{(U_1\ldots U_n) \not\in T_{n,\varepsilon,p}} &= \prob{\bigcup\limits_{u\in\U} B_u}\\
            &\leqslant \sum\limits_{u\in\U} \prob{B_u}
        \end{aligned}
    \]
    
    $B_u$ is the event that $\left\lvert \frac{1}{n} \sum\limits_{i=1}^n \underbrace{\mathbb{1}\set{U_i = n}}_{X_i = \begin{cases}
    1 & \text{ wp } p(u)\\
    0 & \text{ wp } 1-p(u)
    \end{cases}} - p(u) \right\rvert > \varepsilon p(u)$.
    
    ie. $\left\lvert \frac{1}{n} \sum\limits_{i=1}^n X_i - p(u) \right\rvert > \varepsilon p(u)$.
    
    \[
        \prob{B_u} \leqslant \frac{1}{n\varepsilon^2 p_{\min}}
    \]
    where $p_{\min} = \min p(u)$.
\end{proof}

For the size of $T_{n, \varepsilon, p}$ observe first the followings
\begin{itemize}
    \item Fix a $(u_1,\ldots,u_n)$ in $T_{n,\varepsilon, p}$. Note that
    \[
        \begin{aligned}
            \prob{(u_1 \ldots u_n)=(U_1 \ldots U_n)} &= p(u_1)\cdots p(u_n)\\
            &= \prod\limits_{u\in\U}p(u)^{\overbrace{\length{u_1\ldots u_n}_u}^{np(n)(1\pm\varepsilon)}} \geqslant \prod\limits_u p(u)^{np(u)(1+\varepsilon)}\\
            &= \prod\limits_u 2^{np(u)\log p(u)(1+\varepsilon)} = 2^{-n(1+\varepsilon)H(u)}\\
            \Ra 1 &\geqslant \prob{(U_1\ldots U_n)\in T_{n,\varepsilon,p}}\\
            &= \sum\limits_{(u_1\ldots u_n \in T_{n,\varepsilon,p})} \prob{(U_1\ldots U_n) = (u_1\ldots u_n)}\\
            & \geqslant 2^{-n(1+\varepsilon)H(u)}\card{T_{n,\varepsilon,p}}\\
            \Ra \card{T_{n,\varepsilon,p}} \leqslant 2 ^{(1+\varepsilon)nH(U)}
        \end{aligned}
    \]
\end{itemize}




    \section[Universal Source Coding]{Universal Source Coding \---- \textsc{Lempel}-\textsc{Ziv} data compression}
        Inspired by \textsc{Tunstall}, 

\begin{algorithm}
    \DontPrintSemicolon
    Set $\D=\U$\;
    \While{The text is not entirely coded}{
        Associate to each $w \in \D$ a $\ceil{\log\card\D}$-bit binary representation, based on dictionary order.\;
        Parse the next word $w$ from the source sequence using $\D$.\;
        Emit the binary representation of $w$.\;
        $\D \la \D\setminus\set{w} \cup \set{wu\vert u\in\U}$\;
    }
\end{algorithm}

\begin{example}
    $\U = \set{a,b,c}$, source = $bbaacbaa$.
    
    $\D = \set{\underset{\underset{00}{\downarrow}}{a},\underset{\underset{01}{\downarrow}}{b},\underset{\underset{10}{\downarrow}}{c}}$, source = $\textcolor{blue}{b}baacbaa$, output = \textcolor{blue}{01}.
    
    $\D = \set{\underset{\underset{000}{\downarrow}}{a},\underset{\underset{001}{\downarrow}}{ba},\underset{\underset{010}{\downarrow}}{bb},\underset{\underset{011}{\downarrow}}{bc},\underset{\underset{100}{\downarrow}}{c}}$ source = $\textcolor{blue}{ba}acbaa$, output = 01\textcolor{blue}{001}
    
    $\D = \set{\underset{\underset{000}{\downarrow}}{a},\underset{\underset{001}{\downarrow}}{baa},\underset{\underset{010}{\downarrow}}{bab},\underset{\underset{011}{\downarrow}}{bac},\underset{\underset{100}{\downarrow}}{bb},\underset{\underset{101}{\downarrow}}{bc},\underset{\underset{110}{\downarrow}}{c}}$ source = $\textcolor{blue}{a}cbaa$, output = 01001\textcolor{blue}{000}
    
    \bigskip    
    
    At the decoder side, we have $\U = \set{a,b,c}$ and the representation $\underbrace{01}_b\underbrace{001}_{ba}\underbrace{000}_{a}\ldots$
    
    
\end{example}

\chapter{Channel}
    Discrete-time channel, discrete valued input/output.

$x_1,x_2\ldots \ra y_1, y_2\ldots$

$x_i \in \X$, $y_i \in \Y$, $\card{X} < \infty$, $\card{Y} < \infty$

Channel will be described by 

\[
    \prob{\underbrace{Y_i = y}_{\substack{\text{Current}\\\text{output}}} | \underbrace{X_i=x_i}_{\substack{\text{Current}\\\text{input}}}, \underbrace{\ldots, X_1 = x_1,Y_{i-1}=y_{i-1},\ldots, Y_1=y_1}_{\text{Events in the past}}}
\]

A channel is said to be memoryless if
\[
    \prob{Y_i = y | X_i=x_i, \ldots, X_1 = x_1,Y_{i-1}=y_{i-1},\ldots, Y_1=y_1} = \prob{Y_i = y | X_i=x_i} = \PP_i(y\vert x_i)
\]

The channel is a \textsc{Markov} chain.

There is a sequence of events:
\begin{enumerate}
    \item $x_1$ is produced $\prob{x_1}$
    \item $y_1$ is produced $\sim \prob{y_1 \vert x_1}$
    \item $x_2$ is produced $\prob{x_2\vert x_1 (y_1)}$
    \item $y_1$ is produced $\sim \prob{y_2 \vert x_2, x_1, y_1}$
\end{enumerate}

We say that the transmission is performed without feedback if
\[
    \prob{x_i \vert x_{i-1} ldots x_1 y_{i-1}\ldots y_1} = \prob{x_i \vert x_{i-1}\ldots x_1}
\]

\begin{theorem}
    If a memoryless channel is used without feedback, then
    \[
        \prob{y_1\ldots y_n\vert x_1\ldots x_n} = \prob{y_1 \vert x_1} \cdots \prob{y_n \vert x_n}
    \]
\end{theorem}
\begin{proof}
    \[
        \begin{aligned}
            \prob{y_1\ldots y_n\vert x_1\ldots x_n} &= \prob{y_1 \vert x_1} \cdots \prob{y_n \vert x_n}\\
            &= \prob{x_1} \prob{y_1\vert x_1} \prob{x_2\vert x_1 y_1}\prob{y_2\vert y_1, x_1, x_2} \ldots\\
            &= \prob{x_1} \prob{y_1\vert x_1} \prob{x_2\vert x_1}\prob{y_2\vert y_1, x_1, x_2} \ldots\\
            &= \prob{x_1} \prob{y_1\vert x_1} \prob{x_2\vert x_1}\prob{y_2\vert x_2} \ldots\\
            &= \prob{x_1\ldots x_n} \prod\limits_{i=1}^n p(y_i \vert x_i)
        \end{aligned}
    \]
    Thus
    \[
        \prob{y_1\ldots y_n \vert x_1\vert x_n} = \prod\limits_{i=1}^n p(y_i \vert x_i)
    \]
\end{proof}

If the source produces a letter each $\tau_S$ seconds (rate of $\rho_S = \frac{1}{\tau_S}$ letters/s) channel accepts input symbols every $\tau_C$ second (rate of $\rho_C = \frac{1}{\tau_C}$), we better have $\tau_S L = \tau_C n$ (where $L$ letters of the sources are encoded into $n$ symbols).

\begin{proposition}
    For a memoryless channel used without feedback 
    \[
        I(X_1\ldots X_n , Y_1 \ldots Y_n) \leqslant \sum\limits_{i=1}^n I(X_i, Y_i)
    \]
\end{proposition}
\begin{proof}
    \[
        \begin{aligned}
            I(X_1\ldots X_n , Y_1 \ldots Y_n) &= \underbrace{H(Y_1\ldots Y_n)}_{\leqslant \sum\limits_{i=1}^n H(Y_i)} - \underbrace{H(Y_1 \ldots Y_n \vert X_1 \ldots X_n)}_{ \begin{aligned}&= \esp{\log\frac{1}{\prob{Y_1\ldots Y_n\vert X_1\ldots X_n}}}\\&=\esp{\log\frac{1}{\prod\limits_{i=1}^n \prob{Y_i\vert X_i}}}\\&=\sum\limits_{i=1}^n\underbrace{\esp{\log\frac{1}{\prob{Y_i\vert X_i}}}}_{H(Y_i\vert X_i}\end{aligned}}\\
            &\leqslant \sum\limits_{i=1}^n H(Y_i) - H(Y_i \vert X_i)\\
            &=\sum\limits_{i=1}^nI(X_i,Y_i)
        \end{aligned}            
    \]
\end{proof}

\begin{definition}
    Given a discrete, memoryless channel $\set{p(y\vert x)}$, define its capacity as
    \[
        C := \max\limits_{p(x)} I(X,Y)
    \]
\end{definition}

\[
    \text{Source} \ra U_1\ldots U_L \overset{\text{Code}}{\longra} X_1\ldots X_n\overset{\text{Channel}}{\longra} Y_1\ldots Y_n \overset{\text{Decode}}{\longra} V_1\ldots V_L
\]

ie $(U_1\ldots U_l) \--- (X_1\ldots X_n) \--- (Y_1\ldots Y_n) \--- (V_1\ldots V_L)$

\begin{theorem}
    Suppose $U_1U_2\ldots$ is a stationary source with entropy rate $H$, and we have the diagram above then
    \[
        \frac{1}{L} H(U_1\ldots U_L\vert V_1\ldots V_l) \geqslant H - \frac{n}{L} C = H - \frac{\tau_S}{\tau_C} C
    \]
\end{theorem}
\begin{proof}
    \[
        \begin{aligned}
            \frac{1}{L} H(U_1\ldots U_L \vert V_1 \ldots V_L) &= \frac{1}{L} H(U_1\ldots U_L) - \frac{1}{L} I(U_1\ldots U_L,V_1\ldots V_L)\\
            & \geqslant H - \frac{n}{L} C
        \end{aligned}
    \]
\end{proof}

\begin{proposition}
    \[
        \frac{1}{L} \sum\limits_{i=1}^n H(U_i \vert V_i) \geqslant \frac{1}{L} H(U_1\ldots U_L \vert V_1\ldots V_L)
    \]
\end{proposition}

\begin{theorem}[\textsc{Fano}'s inequality]
    Suppose $U, V$ are random variable with the same support. Let $p=\prob{U\neq V}$ then
    \[
        H(U\vert V) \leqslant h_2(p) + p\log(\lvert \U \rvert - 1)    
    \]
    where $h_2(p) = p \log \frac{1}{p} + (1-p) \log\frac{1}{1-p}$ is the binary entropy function.
\end{theorem}
\begin{proof}
    Let $W = [U\neq V]$. $H(W) = h_2(p)$.
    
    \[
        \begin{aligned}
            H(UW \vert V) &= H(U\vert V) + H(W\vert UV)\\
            &= H(W\vert V) + H(I\vert WV)\\
            &\leqslant H(W) + H(U\vert WV)\\
            &=h_2(p) + \prob{X=0}H(U\vert W=0,V) + \prob{X=1}H(U\vert W=1,V)
        \end{aligned}            
    \]
\end{proof}

\begin{definition}[Errors]
    \[
        \begin{aligned}
            P_{e,i} &:= \prob{U_i \neq V_i}\\
            \overline{P_e} &:= \frac{1}{L} \sum\limits_{i=1}^L \prob{U_i \neq V_i}
        \end{aligned}
    \]
\end{definition}


\begin{theorem}[Bad news]
    No matter how the Encoder and Decoder are designed, we have
    \[
        h_2(p) + \overline{P_e} \log(\lvert\U\rvert - 1)    \geqslant \tau_S\left( \frac{H}{\tau_S} - \frac{c}{\tau_C}\right)
    \]
\end{theorem}
In particular, if $\delta := \frac{H}{\tau_S} - \frac{c}{\tau_C}>0$ then $\overline{P_e}$ is bounded away from 0 by some function of $\delta$, regardless of how the encoder and decoder is designed.

$P_{e,\max} = \max\limits_i \prob{U_i \neq V_i} \geqslant \overline{P_e}$.
\[
    \begin{aligned}
        P_{e,block} &= \prob{(V_1\ldots V_l)\neq(U_1\ldots U_L)}\\
        &=\prob{\bigcup\limits_{i=1}^L\set{U_i\neq V_i}}\\
        &\geqslant P_{e,\max}
    \end{aligned}
\]

\begin{proof}[Proof of the theorem]
    Suppose we have Enc and Dec as
    \[
        \begin{aligned}
            (U_1\ldots U_L) \to &Enc \to (X_1\ldots X_n)\\
            (Y_1\ldots Y_n) \to &Dec \to (V_1\ldots V_L)
        \end{aligned}   
    \]
    
    We already prove $\frac{1}{L}\sum\limits_{i=1}^L H(U_i \vert V_i) \geqslant H-\frac{\tau_S}{\tau_C} C$.
    
    We also have proved $h_2(P_{e,i}) + P_{e,i}\log(\lvert \U\rvert -1) \geqslant H(U_i \vert V_i)$.
    
    So, we will be done is we can show
    \[
        h_2(\overline{P_e}) + \overline{P_e}\log(\lvert \U\rvert -1) \geqslant \frac{1}{L} \sum\limits_{i=1}^L (h_2(P_{e,i}) + P_{e,i} \log(\card{\U} -1)
    \]
    equivalent to
    \[
        h_2\left( \frac{1}{L} \sum\limits_{i=1}^L P_{e,i} \right) \geqslant \frac{1}{L} \sum\limits_{i=1}^L h_2(P_{e,i})
    \]
    \begin{lemma}
        \[
            h_2\left( \frac{1}{L} \sum\limits_{i=1}^L p_i \right) \geqslant \frac{1}{L} \sum\limits_{i=1}^L h_2(p_i)
        \]
    \end{lemma}        
    \begin{proof}
        Let $a_1\ldots A_L$ be binary random variables with $\prob{A_i = 0} = 1-p_i$ and $\prob{A_1 = 1} = p_i$, we have $H(A_i) = h_2(p_i)$. Let $B\in \brackets{1,L}$, independent of $(A_1\ldots A_L)$ and $\prob{B=i} = \frac{1}{L}$. Let $A=A_B$. Then 
        \[
            \begin{aligned}
                \prob{A = 1} &= \sum\limits_{i=1}^L\prob{A_i = 1 \vert B=i}\prob{B=i}\\
                &= \frac{1}{L} \sum\limits_{i=1}^L p_i
            \end{aligned}                    
        \]
        $\Ra H(A) = h_2\left( \frac{1}{L} \sum\limits_{i=1}^L p_i \right)$.
        
        Also $H(A\vert B) = \sum\limits_{i=1}^L \underbrace{H(A\vert B=i)}_{\underbrace{H(A_i\vert B=i)}_{H(A_i)}}\prob{B=i} = \frac{1}{L}\sum\limits_{i=1}^L h_2(p_i)$
    \end{proof}
\end{proof}

Given a source $(U_1U_2\ldots), \tau_S$ and a channel $\set{P(y\vert x)}, \tau_C$ Let us say that (source, channel) is compatible if $\forall \varepsilon > 0,\exists Enc, Dec: \overline{P_e} < \varepsilon$.

The theorem we just proved is equivalent to (source, channel) is compatible $\Ra$ $\frac{H}{\tau_S} \leqslant \frac{C}{\tau_C}$.

We now attempt the other direction ($\La$).

Suppose we insist on modular design.

$U_1U_2\ldots$ (stationary, entropy rate $H$)

The channel encoder and channel decoder we seek are of the following kind.
\[
    f:\set{0,1}^k \to \X^n \equiv f:\brackets{1,2^k} \to \X^n\overset{\sim}{\equiv} f:\brackets{1,M} \to \X^n
\]

\begin{definition}
    \[
        \opname{rate}(f) = \frac{k}{n} \equiv \frac{\log M}{n}
    \]
\end{definition}

$\emptyset : \Y^n \to \set{0,1}^k$

\begin{definition}
    \[
        \prob{error\vert i} = \prob{\emptyset(Y_1,\ldots,Y_n) \neq i \vert (X_1\ldots X_n) = f(i))}    
    \]
    \[
        \overline{error}(f,\emptyset,channel) = \frac{1}{M} \sum\limits_{i=1}^M \prob{error\vert i}    
    \]
    \[
        error_{\max}(f,\emptyset,channel) = \max\limits_{1\leqslant i\leqslant M} \prob{error \vert i}
    \]
\end{definition}

\begin{definition}
    Given a channel $\prob{y\vert x}$ we say that $R$ is an achievable rate if $\forall \varepsilon > 0$, there exists $f,\emptyset$ such that $rate(f) \geqslant R$, $error_{\max}(f,\emptyset, P) \leqslant \varepsilon$.
\end{definition}

\begin{theorem}
    Given a channel $\prob{y\vert x}$ with $C= \max_{p_X} I(X,Y)$ then any $R < C$ is achievable.
\end{theorem}

\begin{theorem}[Good news]
    If $\frac{H}{\tau_S} < \frac{C}{\tau_C}$, can achieve $\overline{P_e} \to 0$.
\end{theorem}


\section{The probabilistic method}

$(\exists P_z : \prob{A(z)} > 0)\Ra(\exists z : A(z))$.

$(\exists P_z : \esp{g(z)} \leqslant \alpha)\Ra(\exists z : g(z) \leqslant \alpha)$.

\paragraph{Random coding}

Let $X_i^{(m)}$ be iid $p(x)$.

Let $T^n(\varepsilon) = \set{(x,y) \left\vert \left\lvert \widehat{P}_{\X,\Y}(x,y) - p(x)p(y\vert x) \right\rvert \leqslant \varepsilon p(x)p(y\vert x) \right.}$

\begin{itemize}
    \item Find all $m$ such that $(X^{(m)}, Y) \in T^n(\varepsilon)$
    \item If only one such message $m$, output $m$
    \item Otherwise, output 0.
\end{itemize}


\begin{lemma}
    \begin{itemize}
        \item if $(X,Y) \overset{idd}{\sim} p(x)p(y\vert x)$, then $\prob{(X,Y) \in T^n(\varepsilon)} \to 1$ as $n\to \infty$.
        \item If $(X, Y) \overset{idd}{\sim} p(x)p(y)$ then $\prob{(X,Y) \in T^n(\varepsilon)} \leqslant 2^{-n(I(X,Y)-\delta(\varepsilon))}$, $\delta(\varepsilon) \underset{\varepsilon \to 0}{\longra} 0$
    \end{itemize}
\end{lemma}
\begin{proof}
    \item
    \begin{itemize}
        \item LLN
        \item $D(p\Vert q) = \sum\limits_z p(z) \log\frac{p(z)}{q(z)}$ so $D(P_{XY} \Vert P_Xp_Y) = I(X,Y)$
    \end{itemize}
\end{proof}

\section{Binary Symmetric Channel}

$p(y\vert x) = \begin{cases}
    1-\delta & x = y\\
    \delta & x \neq y
\end{cases}$ with $\X = \Y = \set{0,1}$.


\[
    \begin{aligned}
        H(Y \vert X=0) &= H(Y \vert X=1)\\
        &= h_2(\delta)\\
        &= -\delta \log \delta - (1-\delta) \log (1-\delta)
    \end{aligned}
\]

So

\[
    \begin{aligned}
        I(X,Y) &= H(Y) - H(Y\vert X)\\
        &= H(Y) - h_2(\delta)\\
        &\leqslant 1 - h_2(\delta)
    \end{aligned}
\]

$C = 1-h_2(\delta)$ bits.

\section{Binary Erase Channel}

$\X = \set{0,1}$, $\Y= \set{0,1,e}$.

$p(y\vert x) = \begin{cases}
    1-\varepsilon & y=x\\
    \varepsilon & y=e
\end{cases}$

$e$ for erasure.

\[
    \begin{aligned}
        H(X \vert Y = 0) &= 0
        H(X \vert Y = 1) &= 0
        H(X \vert Y = e) &= H(X)
    \end{aligned}
\]

\[
    \begin{aligned}
        I(X,Y) = (1-\varepsilon) H(X)\\
        &\leqslant 1-\varepsilon
    \end{aligned}
\]

$C = 1-\varepsilon$ bits.

\section{Convexity and Concavity}

$f : D \to \RR$ is convex if blablabla...

Fix $p(y\vert x)$ consider $I(P_x) = I(X,Y) $. 

$D = \set{(P_1,\ldots,P_k) \left\vert \card{\X} = k, \forall i\in\brackets{i,k}, P_i\geqslant 0, \sum\limits_{i=1}^n P_i = 1\right.}$

\begin{theorem}
    $I(P_X)$ est concave.
\end{theorem}

Maximize $f(P_X)$ such that $P(x)\geqslant 0$, $\sum\limits_xP_X(x) = 1$.

$f(P_x) = f(P_1,\ldots,P_k)$. Suppose some $(P_1,\ldots,P_k)$ "believed optimal".

If so, it must be that $f(P_1,\ldots,P_i+\varepsilon,\ldots, P_j-\varepsilon,\ldots,P_k) - f(P_1,\ldots,P_k) \leqslant 0$.

LHS $= \varepsilon \frac{\partial f}{\partial P_i}-\varepsilon\frac{\partial f}{\partial P_j}+o(\varepsilon)$. Should have $\forall(i,j), P_j > 0 \Ra \frac{\partial f}{\partial P_i}-\frac{\partial f}{\partial P_j} \leqslant 0$.In particular $\forall(i,j), P_i>0\wedge P_j > 0 \Ra \frac{\partial f}{\partial P_i}-\frac{\partial f}{\partial P_j} = 0$


\begin{definition}[\textsc{Karush Kuhn Tucker} conditions (KKT)]
    $(P_1,\ldots, P_k)$ satisfies KKT if $\exists \lambda:$
    \[
        \begin{aligned}
            \forall i, P_i>0 &\Ra \frac{\partial f}{\partial P_i} = \lambda\\
            \forall i, P_i=0 &\Ra \frac{\partial f}{\partial P_i} \leqslant \lambda\\
        \end{aligned}    
    \]
\end{definition}

$\approx$ like $\underset{x}{\opname{argmax}} g(x)$ must have $\nabla g(x) = G$ (Unconstrained).

\begin{theorem}
    Consider $\max\limits_{\substack{(P_1,\ldots,P_n)\\\forall i\in\brackets{1,n}, P_i \geqslant 0\\\sum\limits_{i=1}^n P_i = 1}} f(x)$ concave differentiable.
    
    $p(x)$ optimal $\LRa$ KKT.
\end{theorem}
\begin{proof}
    ($\La$): ie. if $(P_1,\ldots,P_k)$ satisfies KKT, then $\forall q, (\forall i\in\brackets{1,n},Q_i \geqslant 0\wedge \sum\limits_{i=1}^n Q_i = 1) \Ra f(q) \leqslant f(p)$.
    
    \[
        \begin{aligned}
            f(\theta q + (1-\theta)p) &\geqslant (1-\theta)f(p) + \theta f(q)\\
            f(\theta q + (1-\theta)p) - f(p) &\geqslant -\theta f(p) + \theta f(q)\\
            f(q) - f(p) &\leqslant \frac{1}{\theta}\left( f(p+\theta(q-p)) - f(p) \right)\\
            \to f(q) - f(p) &\leqslant \sum\limits_i (q_i - p_i) \frac{\partial f}{\partial P_i}\\
            &\leqslant \sum\limits_i(q_i-p_i)\lambda\\
            &= \lambda \left(\sum\limits_i q_i - \sum\limits_i p_i\right)\\
            &= 0
        \end{aligned}            
    \]
\end{proof}

\begin{theorem}
    \[
        p(x) \in \opname{argmax} I(X,Y) \LRa \exists \lambda: \forall x, \sum\limits_y p(y\vert x) \log \frac{p(y\vert x)}{p(y)} \leqslant \lambda
    \]
    with equality when $p(x) > 0$. If so, then $\lambda = C$
\end{theorem}
$\approx$ each used input reveals the same amount of information.
\begin{proof}
    Use KKT, prove $\frac{\partial I(X,Y)}{\partial P(x)} = \sum\limits_y p(y\vert x) \log \frac{p(y\vert x)}{p(y)}-\log e$. (careful: $p(y)$ depends on $p(y)$). $I(X,Y) = H(Y) - H(Y\vert X)$.

    \bigskip
    
    ($\lambda = C$): sum $p(x) \sum\limits_y p(y\vert x) \log \frac{p(y\vert x)}{p(x)} = \lambda p(x)$ over $x$.    
\end{proof}

\begin{remark}
    For symmetric channels, can "guess" $P_x = \left( \frac{1}{K},\ldots,\frac{1}{K}\right)$ and verify KKT.
\end{remark}

\begin{definition}[Z-channel]
    \[
        \begin{aligned}
            p(0 \to 0) &= 1\\
            p(1 \to 0) &= \delta\\
            p(1 \to 1) &= 1 - \delta
        \end{aligned}
    \]
\end{definition}

\[
    \begin{aligned}
        p_Y(0) &= 1-P_X(1)(1-\delta)\\
        P_Y(1) &= P_X(1)(1-\delta)
    \end{aligned}            
\]

KKT:
\begin{itemize}
    \item $x=0$: $\lambda = \log \frac{1}{P_Y(0)}$
    \item $x=1$: $\lambda = \delta \log \frac{\delta}{P_Y(0)} + (1-\delta) \log \frac{1-\delta}{P_Y(1)}$
\end{itemize}


\[
    \begin{aligned}
        h_2(\delta) &= -\delta \log \delta - (1-\delta) \log(1-\delta)\\
        \lambda &= -h_2(\delta) + \delta \lambda - (1-\delta) \log\left (1-2^{-\lambda}\right )\\
        \lambda &= \log\left( 1-2^{-\frac{h_2(\delta)}{1-\delta}} \right) = C\\
        P_Y(0) &= 2^{-\lambda}\\
        &= \frac{1}{1+2^{\frac{-h_2(\lambda)}{1-\delta}}}\\
        P_Y(1) &= 1- P_Y(0)\\
        P_X(1) &= \frac{P_Y(1)}{1-\delta}\\
        &= \frac{1}{(1-\delta)\left(1+2^{\frac{h_2(\delta)}{1-\delta}}\right)}\\
        P-X(0) &= 1-P_X(1)
    \end{aligned}
\]

\section{Continuous channels}

So far: discrete (memoryless) channels.

But we would to add a Gaussian noise. So, we need a real-valued channel.

Let $(X,Y)$ have probability density functions $f_X, f_Y, f_{XY}, f_{Y\vert X}$ etc..

\[
    \begin{aligned}
        h(X) &= \int f_X(x) \log f_X(x) \dd x\\
        &= \esp{-\log f_X(X)}\\
        h(X\vert Y) &= \esp{-\log f_{X\vert Y} (X\vert Y)}\\
        h(X,Y) &= \esp{-\log f{XY}(X,Y)}\\
        D(f_X \Vert g_X) &= \EE_f\left[\log\frac{f_X(X)}{g_X(X)}\right]\\
        &= \int f_X(x) \log \frac{f_X(x)}{g_X(x)} \dd x\\
        I(X,Y) &= D(f_{XY} \Vert f_X f_Y)\\
        &=\iint f_{XY}(x,y) \log \frac{f_{Y\vert X}(y,x)}{f_Y(y)} \dd x\dd y
    \end{aligned}
\]

\begin{enumerate}
    \item $I(X,Y) = h(X) - h(X\vert Y) = h(Y) = h(Y\vert X)$ \textsc{Bayes} ok
    \item $D(f\Vert g) \geqslant 0$ with equality $\LRa$ $f = g$
    \item $I(X,Y) \geqslant 0$ with equality $\LRa$ $X \perp\hspace{-7pt}\perp Y$
    \item Conditioning reduces $h$: $h(X\vert Y) \leqslant h(X)$
    \item Chain rules: $h(X_1,\ldots, X_n) = h(X_1) + h(X_2 \vert X_1) + \cdots + h(X_n \vert X_1,\ldots,X_{n-1})$
    \item $h(X_1,\ldots,X_n) \leqslant \sum\limits_{i=1}^n h(X_i)$
\end{enumerate}

But

$h(X) \not\geqslant 0$ in general.

\begin{example}
    $X \sim \U\left( \left[ \frac{-a}{2},\frac{a}{2} \right] \right)$.
    
    \[
        \begin{aligned}
            h(X) &= \log_2 a\\
        \end{aligned}
    \]
\end{example}

\begin{proposition}
    \[
        \begin{aligned}
            \forall c \in \RR, &h(X + c) = h(X)\\
            &h(cX) = \log \lvert c \rvert + h(X)
        \end{aligned}
    \]
\end{proposition}

\begin{definition}[Entropy typical set]
    \[
        T^n(\varepsilon) = \set{x \left\vert \left\lvert \frac{1}{n} \log f_X(x_i) - h(X) \right\rvert \leqslant \varepsilon \right.}
    \]
\end{definition}

\begin{proposition}
    Discrete case
    \[
        \card{T^n(\varepsilon)} = 2^{n(H(X) + \delta(\varepsilon))}
    \]

    Continuous case
    \[
        Vol(T^n(\varepsilon)) = 2^{nh(X) + \delta(\varepsilon)}
    \]
    where $\delta \underset{\varepsilon \to 0}{\to} 0$.
\end{proposition}

\begin{proposition}
    \[
        \prob{X \in T^n(\varepsilon)} \to 1
    \]
\end{proposition}

Let $X$ be continuous rv with a continuous $f_X(x)$ and $X_\Delta = \Delta_i$ for $\Delta i \leqslant X < \Delta(i+1)$.

\begin{theorem}
    \[
        \lim\limits_{\Delta \to 0}( H(X_\Delta) + \log \Delta) = h(X)
    \]
\end{theorem}
\begin{proof}
    \[
        \begin{aligned}
            p_i &= \prob{X_\Delta = \Delta_i}\\
            &= \int_{i\Delta}^{(i+1)\Delta} f(x) \dd x\\
            &= f(x_i) \Delta \text{\qquad for some }x_i\in [i\Delta, (i+1)\Delta]
        \end{aligned}
    \]
    
    \[
        \begin{aligned}
            H(X_\Delta) &= - \sum\limits_i p_i\log p_i\\
            &= - \sum\limits_i \Delta f(x_i) \log (\Delta f(x_i))\\
            &= \underbrace{- \sum\limits_i \Delta f(x_i) \log f(x_i)}_{\scriptscriptstyle \begin{aligned}\underset{\Delta \to 0}{\to} &-\int f(x) \log f(x)\dd x \\ &h(X) \end{aligned}} - \underbrace{\underbrace{\sum\limits_i f(x_i) \Delta}_{\int f(x) \dd x = 1} \log \Delta}_{\log \Delta}
        \end{aligned}            
    \]
\end{proof}

\begin{theorem}
    \[
        I(X_\Delta, Y_\Delta) \underset{\Delta\to 0}{\to} I(X,Y)
    \]
\end{theorem}
\begin{proof}
    \[
        \begin{aligned}
            I(X_\Delta, Y) &= H(X_\Delta) - H(X_\Delta \vert Y)\\
            &= (h(X - \log \Delta + o(\Delta)) - (h(X\vert Y) - \log \Delta + o(\Delta))\\
            &= h(X) - h(X\vert Y) + o(\Delta)\\
            &\to I(X,Y)
        \end{aligned}
    \]
    
    So $I(X_\Delta, Y) \to I(X,Y)$. Similarly $I(X, Y_\Delta) \to I(X,Y)$ and $I(X_\Delta, Y_\Delta) \to I(X,Y_\Delta)$. So $I(X_\Delta, Y_\Delta) \to I(X,Y)$.
\end{proof}

\subsection{Gaussian Noise}

$X \sim \N(0,\sigma^2)$.

\[
    f_X(x) = \frac{1}{\sqrt{2\pi\sigma^2}} e^{\frac{-x^2}{2\sigma^2}}
\]

\[
    \begin{aligned}
        h(X) &= \esp{-\ln f_X(x)}\\
        &= \esp{\frac{1}{2} \ln(2\pi\sigma^2) + \frac{1}{2\sigma^2}X^2}\\
        &= \frac{1}{2}\left( 1+\ln(2\pi\sigma^2) \right)\\
        &= \frac{1}{2} \ln(2\pi e\sigma^2)\text{ NATS}\\
        &= \frac{1}{2} \log(2\pi e\sigma^2) \text{ Bits}
    \end{aligned}
\]

\begin{theorem}[Maximal entropy]
    If $\Var{X} = \sigma^2$, then $h(X) \leqslant \frac{1}{2} \log (2\pi e\sigma^2)$ with equality iff $X\sim \N(\mu,\sigma^2)$ for some $\mu$.
\end{theorem}
\begin{proof}
    Wlog $\esp{X} = 0$.
    
    Let $X$ have pdf $f$ and let $g$ be the $N(0,\sigma^2)$ pdf.
    
    \[
        \begin{aligned}
            \int f(x) \log g(x) \dd x &= \EE_f\left[ c_1 + c_2 X^2 \right]\\
            &= \EE_g\left[ c_1 + c_2 X^2\right]\\
            &= \int g(x) \log g(x) \dd x
        \end{aligned}
    \]
    
    Goal: show $-\int f(x) \log f(x) \dd x\leqslant -\int g(x) \log g(x) \dd x \LRa -\int f(x) \log f(x) \dd x\leqslant -\int f(x) \log g(x) \dd x$. Ie. $\int f(x) \log \frac{f(x)}{g(x)} \geqslant 0$ ie. $\D(f\Vert g) \geqslant 0$.
\end{proof}

\subsection{Vector Extensions}
\begin{itemize}
    \item If $X \in \RR^d$, $X=\N(0, K)$ where $K \in \M_d(\RR)$, then $h(X) = \frac{1}{2}\log \det(2\pi e K)$.
    \item $\opname{Cov}(X)) = K$, then $h(X) \leqslant \frac{1}{2} \log \det (2\pi e K)$.
    \item $h(X+\varepsilon = h(X)$.
    \item $h(AX) = h(X) + \log \lvert\det A \rvert$.
\end{itemize}

\subsection{Additive White Gaussian Noise Channel}

$Y = X + Z$, $Z\sim \N(0,\sigma^2)$. White noise: $Z_i \perp\hspace{-7pt}\perp Z_j$ for $i\neq j$ (ie. memoryless).

\[
    \begin{aligned}
        I(X,Y) &= I(X+Z) - \underbrace{h(X+Z \vert X}_{\scriptscriptstyle\begin{aligned}
                &= h(Z + X)\\&=h(Z)\\&=\frac{1}{2}\log(2\pi e \sigma^2))
        \end{aligned}}
    \end{aligned}
\]

Cost function $b(x)$.

Average constraint $\frac{1}{M} \sum\limits_{n=1}^{M} \frac{1}{n} \sum\limits_{i=1}^n b(x_i^{(u)}) \leqslant P$.

\begin{theorem}
    If $\frac{H}{\tau_S} > \frac{C(P)}{\tau_C}$, any code  with $\sum\limits_u p(u) \frac{1}{n} \sum\limits_{i=1}^nb(x^{(u)}_i) \leqslant P$ has bit error rate $P_e \not\to 0$.
\end{theorem}

\begin{lemma}
    \[
        \sum\limits_{i=1}^n I(X_i, Y_i) \leqslant , C(P)
    \]
\end{lemma}

\subsection{Parallel Gaussian Channel}

Let $x = (X_1,\ldots,X_k)$.

$Z_1 \sim\N(0,\sigma_1^2)$ and  $Z_k \sim\N(0,\sigma_k^2)$

$Y_i = X_i + Z_i$.

$Z_i$ pairwise independent.

Power constraint

$b(x) = b(x_1,\ldots,x_n) = \sum\limits_{i=1}^k x_i^2$

Constrain $\esp{b(X)} \leqslant P$.

$C(P) = \max\limits_{\sum\limits_{i=1}^k \esp{X_i^2}} I(X_1,\ldots,X_k ; Y_1,\ldots, Y_k) = \max\limits_{\substack{P_1,\ldots,P_k\\P_i \geqslant 0\\\sum\limits_{i=1}^k P_i \leqslant P}} \sum\limits_{i=1}^n\frac{1}{2} \log\left(1+\frac{P_i}{\sigma_i^2}\right)$.

Let $\alpha_i=\frac{P_i}{P}$ (proportion of power used in subchannel $i$)

KKT : 
\[
    \begin{aligned}
        \text{Optimal} &\LRa \exists \lambda: \frac{\partial f}{\partial \alpha_i} \leqslant\lambda\\
        &\LRa \exists \lambda : \sigma_i^2+P\alpha_i \geqslant \lambda\\
        &\LRa \exists \lambda \sigma_i^2+P_i \geqslant \lambda \text{ with equality if }P_i > 0
    \end{aligned}
\]
$P_i = \max\set{0,\lambda - \sigma_i^2}$

\section{Build Your Own Encoder/Decoder}

For a while, we will be concerned with the Binary Symmetric Channel  ($\prob{x=y} = p$ et $\prob{x=1-y} = 1-p$).

Code for the BSC is a array $M (=2^{nR}) \times n$ of binary entries.

Given $\overline{y} = (y_1,\ldots,y_n)$ find the codeword (a table row) closet to $\overline{y}$ in \textsc{Hamming} distance.

Why?
\[
    \begin{aligned}
        \prob{\overline{y}\vert \overline{x}} &= \prod\limits_{i=1}^n \underbrace{\prob{y_k \vert x_i}}_{\begin{cases}1-p & y_i = x_i\\p & y_i \neq x_i\end{cases}}\\
        &= (1-p)^n \left( \frac{p}{1-p} \right)^{\card{\set{i}{y_i\neq x_i}}}
        &= \text{decreasing in }d_H(\overline{x},\overline{y})
    \end{aligned}
\]
So
\[
    \text{Maximum likehood} \LRa \text{Minimum distance}
\]

\begin{definition}[\textsc{Hamming} distance]
    \[
        d_H(\overline{x}, \overline{y}) := \card{\set{i\vert y_i\neq x_i}}
    \]
\end{definition}

We immediately see $(n, nR)$ as visible parameters of the code (named $\C$). Let us also add $d = d_H(\C) = \min\limits_{\substack{1\leqslant i\leqslant M\\1\leqslant j < i}} d_H\left(X^{(i)}, X^{(j)}\right)$. as the minimum distance of $\C$ to these parameter.


\begin{theorem}[\textsc{Singleton} bound]
    \[
        M > 2^k \Ra d \geqslant n - k
    \]
    where $M$ is the number of codewords.
\end{theorem}
\begin{proof}
    Given $k$ columns, we can only write $2^k$ different things. So, is we have to write more than $2^k$ codewords, at least 2 will agree on these $k$ columns. So, their \textsc{Hamming} distance is at most $n - k$.
\end{proof}

\begin{corollary}[\textsc{Singleton} bound, other style]
    \[
        M = 2^k \Ra d \geqslant n - k + 1
    \]
    where $M$ is the number of codewords.
\end{corollary}

\begin{theorem}
    The \textsc{Hamming} distance is a distance.
\end{theorem}
\begin{proof}
    Isn't it obvious?
\end{proof}

\begin{corollary}
    Suppose $(x, x') \in \left(\set{0,1}^n\right)^2$ with $d = d_H(x, x')$. Set $r = \floor{ \frac{d-1}{2}}$. Then $\mathcal{B}(x, r) \cap \mathcal{B}(x',r) = \emptyset$.
\end{corollary}
\begin{proof}
    Suppose not, then $\exists \overline{y} : d_H(x,y) \leqslant r \wedge d_H(x',y) \leqslant r$. But $d_H(x,y) \leqslant r \Ra d_H(x',y) \leqslant r \Ra d(x,x') \leqslant 2r < d$. Contradiction.
\end{proof}

Also note $\card{\mathcal{B}(x,r)} = \sum\limits_{k=0}^{\min(n,r)}\binom{n}{k}$.

\begin{theorem}[Sphere packing bound]
    Suppose $\C \subseteq \set{0,1}^n$ is a binary code with $M$ codeword. Then
    \[
        M \sum\limits_{k=0}^{\min\left(n,\ceil{\frac{d-1}{2}}\right)}\binom{n}{k} \leqslant 2^n
    \]
\end{theorem}
\begin{proof}
    By the corollary, the balls $\set{\mathcal{B}(x,r)\vert x \in \C}$ are all disjoint. Then
    \[
        \sum\limits_{x\in \C}\card{\mathcal{B}(x,r)} = \card{\bigcup\limits_{x \in \C} \mathcal{B}(x,r)}
    \]

    But $\bigcup\limits_{x \in \C} \mathcal{B}(x,r) \subseteq \set{0,1}^n$ so
    
    \[
        \card{\bigcup\limits_{x \in \C} \mathcal{B}(x,r)} \leqslant 2^n
    \]
\end{proof}


\begin{theorem}[\textsc{Gilbert-Varshamov} bound]
    Given $n, d$, there exists a code $\C$ with blocklength $n$, $d_{\min} \geqslant d$
    \[
        M \sum\limits_{k=0}^{d-1} \binom{n}{k} \geqslant 2^n
    \]
\end{theorem}
\begin{proof}
    We take a word in $\set{0,1}^n$ and we mark the ball of radius $d-1$. And we redo with an unmarked word until the whole space is marked.
    \[
        \underbrace{M}_{\text{\# of codeword}} \underbrace{\sum\limits_{k=0}^{d-1} \binom{n}{k}}_{\geqslant \text{\# of word marked at each step}} \geqslant \underbrace{2^n}_{\text{total number of words}}
    \]
\end{proof}

\begin{definition}
    A code $\C\subseteq\set{0,1}^M$ is said to be linear if $\forall (x,y)\in\C^2, x+y\in\C$.
\end{definition}

\begin{example}
    $\C = \set{001,111}$ is not linear ($000$ missing).
    
    $\C = \set{001,111,000,110}$ is linear since 
    
    \[
        \begin{aligned}
            \C&=\opname{Vect}_{\ZZ/2\ZZ}((001,111))\\
            &= \set{\left(\begin{matrix}u_1&u_2\end{matrix}\right)\left(\begin{matrix}0&0&1\\1&1&1\end{matrix}\right)\vert (u_1,u_2)\in\set{0,1}^2}\\
            &= \set{\left(\begin{matrix}0&1\\0&1\\1&1\end{matrix}\right)\left(\begin{matrix}u_1\\u_2\end{matrix}\right)\vert (u_1,u_2)\in\set{0,1}^2}
        \end{aligned}
    \]
\end{example}

\begin{proposition}
    If $\C$ is a linear code, then $\card{\C} = 2^k$ for some integer $k$, and there exists a $k\times n$ binary matrix $G$ such that
    \[
        \C = \M_{1,k}(\ZZ/2\ZZ)G    
    \]
\end{proposition}

\begin{proposition}
    If $\C\subseteq \set{0,1}^M$ is a linear code whith $\card{\C} = 2^k$, then there exists a $n\times(n-1)$ matrix $H$ such that $\C = \opname{Ker} H$.
\end{proposition}

$H$ is called the parity-check matrix.

\begin{definition}
    The \textsc{Hamming} weight $w_H(x)$ of a vector $x$ is the sum of its components. Ie. it is the 1-norm.
\end{definition}

\begin{definition}
    Given a linear code $\C$, let $w_{\min}(\C) = \min_{\substack{x\in\C\\x\neq 0}w_H(x)}$.
\end{definition}

\begin{theorem}
    For a linear code $\C$, $w_H(\C) = d_{\min}(\C)$
\end{theorem}
\begin{proof}
    \[
        d_H(x,0) = w_H(x)
    \]
    so
    \[
        x_H(\C) \geqslant d_{min}(C)
    \]
    Also if $x\neq x' \in \C$, $w_H(x+x') = d_H(x,x')$.
\end{proof}

\begin{example}
    Let
    \[
        H = \left(
            \begin{matrix}
                1 & 0 & 0\\
                0 & 1 & 0\\
                0 & 0 & 1\\
                1 & 1 & 0\\
                0 & 1 & 1\\
                1 & 0 & 1\\
                1 & 1 & 1\\
            \end{matrix}
        \right)
    \]
    
    Let $\C = \opname{Ker} H^T$.
    
    \[
        \left(
        \begin{matrix}
        x_1\\x_2\\x_3
        \end{matrix}
        \right) = \left( \begin{matrix}
        1&0&1&1\\
        1&1&0&1\\
        0&1&1&1\\
        \end{matrix} \right) \left( \begin{matrix}
        x_4\\x_5\\x_6\\x_7
        \end{matrix}\right)
    \]
    So $\card{\C} = 16$.
    The spheres of radius 1 around the 16 codewords are disjoint and completely cover $\set{0,1}^7$.
\end{example}

We can generalize to $(2^m-1,2^m-m-1,3)$ \textsc{Hamming} code.















    
\chapter{Rate-Distortion theory}

Rate-distortion theory = Lossy compression = Lossy source coding = Quantization.

Setup
\begin{itemize}
    \item Memory less stationary source: $\U_1,\U_2,\ldots$ iid $\sim p_u$, $\U_i \in \U$
    \item Reproduction alphabet $\V$
    \item Distortion measure $d:\U\times\V \to \RR^+$
\end{itemize}

What we want

$U_1,\ldots,U_n \overset{f}{\to} nR \text{ bits integer} \overset{\Phi}{\to} V_1\ldots,V_n$

Two parameters:
\begin{itemize}
    \item $R$: bits/source letter
    \item $D = \esp{\frac{1}{n} \suml_{i=1}^n d(U_1, V_1) }$: expectation average distribution.
\end{itemize}

Question: what $(R,D)$ pairs are feasible?

\begin{theorem}[Bad news]
    Suppose $\U^n := U_1,\ldots,U_n \overset{f}{\to} nR \text{ bits integer} \overset{\Phi}{\to} V_1\ldots,V_n$ with expected average distribution $=D$, then $\exists p_(v\vert u) : R \geqslant I(U,V) \wedge D \geqslant \esp{d(U,V)}$, computed wrt $p_{uv}(u,v) = p_u(u)p(v\vert u)$.
\end{theorem}
\begin{proof}
    \[
        \begin{aligned}
            n R &\geqslant I(\U^n, \V^n)\\
            &= H(U^n) - H(U^n\vert V^n)\\
            &= \suml_iH(U_i) - \sum_i H(U_i\vert V^nU^{i-1})\\
            &\geqslant \suml_i (H(u_i) - H(U_i |vert V_i)\\
            &= \suml_{i=1}^n I(U_i, V_i)
        \end{aligned}
    \]
    
    \begin{enumerate}
        \item $R \geqslant \frac{1}{n} \suml_{i=1}^n I(U_i, V_i)$
        \item $D=\frac{1}{n} \suml_{i=1}^n \esp{d(U_1, V_i)}$
    \end{enumerate}
    
    denote the distribution of $(U_1, V_i)$ by $p_i(u,v)$ note that $p_i(u,v) = p_u(u) p_i(v\vert u)$.
    
    Consequently 
    \begin{itemize}
        \item $R \geqslant \frac{1}{n} \suml_{i=1}^n I(U,V) \vert_{p_{uv} = p_u(u) p_i(v\vert u)}$
        \item $D = \frac{1}{n} \suml_{i=1}^n \esp{d(U,V)} \vert_{p_{uv} = p_u(u) p_i(v\vert u)} = \suml_{u,v} \frac{1}{n} \suml_{i=1}^n d(u,v) p_u(u) p_i(v\vert u) = \esp{d(U,V)}$
    \end{itemize}

    Also note
    \[
        \frac{1}{n} \suml_{i=1}^nI(U,V)\vert_{p_{uv} = p_u(u)p_i(v\vert u)} = I(U,V \vert Q)
    \]
    where $(Q,U,V)$ has the distribution $\prob{Q=i, U=u, V=v} = \frac{1}{n} = \frac{1}{n} p_0(u) p_i(v\vert u)$

    $\Ra$ $\prob{U=u, V=v} = p_u(u) \frac{1}{n} \suml_{i=1}^n p_i(v\vert u) = p(u,v)$ and $I(U,V \vert Q) = I(U,VQ) \geqslant I(U,V)$
    
    So we see that for $(v\vert u) = \frac{1}{n} \sum p_i(v\vert u)$ we have
    \begin{enumerate}
        \item $R\geqslant I(U,V)$
        \item $D = \esp{d(U,V)}$
    \end{enumerate}
    as we were supposed to show.
\end{proof}

\begin{theorem}[Good news]
    Given $p_u(u)$ (ie. fiven a memoryless stationary source) and given $p(v\vert u)$ given any $\varepsilon > 0$ then, there exists $n,f,\Phi$ such that with $U_1,\ldots,U_n \overset{f}{\to} nR \text{ bits integer} \overset{\Phi}{\to} V_1\ldots,V_n$ with
    \[
        \begin{aligned}
            R &\leqslant I(U,V) + \varepsilon\\
            D &\leqslant \esp{d(U,V)} + \varepsilon
        \end{aligned}    
    \]
\end{theorem}
\begin{proof}
    Given $p(v\vert u)$, given $R = I(U,V) + \varepsilon$ generate $2^{nR}$ sequences $V^n(1),\ldots, V^n(2^{nR})$. By choosing $\set{ V_i(m) \vert 1\leqslant i \leqslant n, 1\leqslant m \leqslant 2^{nR}}$ iid $\sim p_v(v)$. And set $\Phi(i) = V^n(i)$
    
    \[
        f(u_1\ldots u_n) = \begin{cases}
            i & \text{if } ((u_1, V_1(i)),\ldots, (u_n, V_n(i))) \in T^n_\varepsilon(p_{uv})\\
            1 & \text{otherwise}
        \end{cases}
    \]
    
    Observations: the rate of this design is $R$, then only worry is about the distortion.
    
    if the success branch happens, then $U^n \overset{f}{\to} \overset{\Phi}{\to} V^n$ with $(U^n,V^n) \in T_\varepsilon^n(p_{uv})$.
    
    In this case 
    \[
        \begin{aligned}
            \frac{1}{n} \suml_{i=1}^n d(U_1, V_i) &= \frac{1}{n} \sum_{u,v} d(u,v) \underbrace{\card{\set{i\vert (U_i, V_i) = (u,v)}}}_{\leqslant n(1+\varepsilon)p_{uv}(u,v)}\\
            &= (1+\varepsilon)\esp{d(U,V)}
        \end{aligned}
    \]
    
    In case of failure,
    \[
        \begin{aligned}
            \frac{1}{n} \suml_{i=1}^n d(U_i, V_i) &\leqslant \maxl_{u,v} d(u,v)\\
            &= D_{\max}
        \end{aligned}
    \]
    
    \[
        \begin{aligned}
            \esp{\frac{1}{n}\sum d(U_i, V_i)} &\leqslant 1(1+\varepsilon) \esp{d(U,V)}+\prob{\text{failure}} D_{\max}
        \end{aligned}    
    \]
    We will show that $\prob{\text{failure}} \overset{n\to\infty}{\longra} 0 $, which will prove the theorem.
    
    Failure happens if $\bigwedge\limits_{i=1}^{2^{nR}}(U^n, V^n(i)) \not\in T_\varepsilon^n(p_{uv})$.
    
    Suppose we fix $U^n = u^n$.
    \[
        \prob{\text{failure} \vert U^n = u^n} = \prodl_{m=1}^{2^{nR}} \prob{(U^n, V^n(m)) \not\in T_\varepsilon^n(p_{uv})} = \left( 1-\prob{(U^n, V^n(m)) \in T_\varepsilon^n(p_{uv})} \right)^{2^{nR}}
    \]
    
    Consider two cases
    
    \begin{enumerate}
        \item $u^n \in T^n_\varepsilon(p_u \Ra \prob{\text{failure}} = \prob{\text{failure} \vert U^n\in T_\varepsilon^n(p_{uv})}$
        \item $u^n \not\in T^n_\varepsilon(p_u \Ra \prob{\text{failure}} = 1$
    \end{enumerate}
    
    Consider the experiment $U_1, \ldots, U_n$ iid $\sim p_u$, $V_1,\ldots, V_n$ iid $\sim p_v$.
    
    \[
        \prob{(U^v,V^n)\in T_{\varepsilon}^n(p_{uv}} \approx 2^{-nI(U,V)(1\pm \varepsilon)}
    \]
    when $u^n \in T^n_\varepsilon(p_u)$ we have
    \[
        \prob{(u^n, V^n(1))\in T^n_\varepsilon(p_{uv}} \geqslant 2^{-nI(U,V)(1+\varepsilon)}
    \]
    \[
        \begin{aligned}
            \prob{\text{failure} \vert u^n} &\leqslant \left(1-2^{-nI(U,V)(1+\varepsilon}\right)^{2^{nR}}\\
            &\leqslant \exp\left(-2^{-nI(U,V)(1+\varepsilon)}\right) 2^{nR}\\
            &= \exp\left(-2^{n\left(R-I(U,V)(1+\varepsilon)\right)}\right)
        \end{aligned}
    \]
    So
    \[
        \begin{aligned}
            \prob{\text{failure}} &= \prob{\text{failure} \vert U^n \in T(p_u)}\prob{U^n \in T(p_u)} + \prob{\text{failure} \vert U^n \not\in T(p_u)}\prob{U^n \not\in T(p_u)}\\
            &\leqslant \prob{\text{failure} \vert U^n \in T} + \prob{U^n\not\in T}
        \end{aligned}
    \]
    So $\prob{\text{failure}} \overset{n\to\infty}{\longra} 0$
\end{proof}

\begin{example}
    $U\in\set{0,1}$ equal prob. $\V = \set{0,1}$, $d(u,v) = [u\neq v]$
    \[
        \begin{aligned}
            R &= 1-h_2(p)\\
            D &= p\\
            0 &\leqslant p \leqslant \frac{1}{2}
        \end{aligned}
    \]
\end{example}

\begin{example}
    $U\sim \N(0,\sigma^2) \in \RR$, $\V \in \RR$, $d(u,v) = (u-v)^2$
    
    \[
        R = \frac{1}{2} \log\frac{\sigma^2}{D}\qquad D \leqslant \sigma^2
    \]
\end{example}


\bibliographystyle{alpha}
\bibliography{bibliography}
\nocite{*}

\end{document}





