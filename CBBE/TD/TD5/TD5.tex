\input{../../../Commun/macros_TD.tex}

\title{Cellular Biology \& Biochemistry for Engineers\\TD5}
\author{Harald \textsc{Hirling}}
\date{29 sept. 2015}

\begin{document}
\maketitle

\section*{46)}

T is probably subject to a feedback inhibition by the compound $C$.

\section*{47)}
(c) $\Ra$ no quaternary structure $\Ra$ smaller protein $\Ra$ slower.

\section*{48)}
They can be metabolised to produce ATP to allow other cells to live.

\section*{49)}


\section*{50)}
(d)

\section*{51)}


\section*{52)}
\begin{verbatim}
5' - GCATTCGTGGGTAG - 3'
3' - CGTAAGCACCCATC - 5'
\end{verbatim}

\section*{53)}
They allow the DNA to wrap around them to save space. They can protect DNA, too.

\section*{54)}


\section*{55)}
\begin{enumerate}
    \item 5' end
    \item Phosphate
    \item Sugar
    \item 3' end
    \item Hydrogen bound
    \item Base
\end{enumerate}

\section*{56)}


\end{document}