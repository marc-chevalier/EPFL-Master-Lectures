\input{../../../Commun/macros_TD.tex}

\title{Cellular Biology \& Biochemistry for Engineers\\TD4}
\author{Harald \textsc{Hirling}}
\date{29 sept. 2015}

\begin{document}
\maketitle

\section*{35)}
The ligand must have the same shape than the site and with the structure convenient to make binds (hydrogen, disulfure...) where the site can.

That means that the ligand X have an high affinity for the protein Y and that other ligand have a way lower affinity.

\section*{36)}
Phosphorylation modify the structure of the protein. Thus, that change their enzymatic activity, that allow association with other protein... 

\section*{37)}
No, the turn-over speed will be higher for A than for B.

\section*{38)}
\subsection*{A.}
\[
    \Delta G = \Delta G^0 + 0.616 \ln \frac{[Y]}{[X]}
\]
because $RT = 0.616$.

1Bar, 37°C, 1M, pH7.

\subsection*{B.}
(a) and (c) will not spontaneously occur.

\subsection*{C.}
The reaction (d), if it is coupled to (a) or (c) will make them happen.

\section*{39)}
Maybe, the receptor has a shape such that he seems to be activated even when GTP is not there.

\section*{40)}
(a)

\section*{41)}
The metabolic pathway in presence of oxygen produce more ATP for each glucose molecule.

\section*{42)}
\subsection*{A.}
Protein kinases.

\subsection*{B.}
They phosphoryl a protein $(\Delta G > 0)$ by using the reaction ATP $\ra$ ADP ($\Delta G < 0$).

\subsection*{C.}
Phosphatase

\section*{43)}
\begin{enumerate}
    \item Glycolysis: cytosol
    \item Citric acid cycle: mitochondria
    \item Conversion of pyruvate to activated acetyl group: mitochondria
    \item Oxidation of fatty acids to acetyl CoA: mitochondria
    \item Release of fatty acids from triacylglycols: mitochondria
    \item Oxidative phophorylation that produce ATP: mitochondria
\end{enumerate}

\section*{44)}
In $CO_2$ and $H_2O$. From the sugar. The will take by plants by photosynthesis to make $O_2$ and sugar.

\section*{45)}
Aerobic phase release $CO_2$. Anaerobic phase is alcoholic fermentation. To start the second phase, we have to stop respiration.

The aerobic phase allow the multiplication of the yeast. As they produce more ATP, they grow up more quickly.

Then, we want an anaerobic phase to produce alcohol thanks to the important population of yeast.

\end{document}