\input{../../../Commun/macros_TD.tex}

\title{Cellular Biology \& Biochemistry for Engineers\\TD7}
\author{Harald \textsc{Hirling}}
\date{3 nov. 2015}

\begin{document}
\maketitle

\section*{68)}
\subsection*{a)}
The translation will begin later. The begin of the protein will be missing.

\subsection*{b)}
The nucleotides will be wrongly grouped in codons. The whole protein will be wrong.

\subsection*{c)}
One amino acid will be wrong.


\section*{69)}
Tyrosine (Y)

\section*{70)}
Denature protein?

\section*{71)}
If a mRNA is durable, it will be translated a lot of times.

\section*{72)}
Side chain of lysine is positively charged but acetylation remove this charge. So histones will not hold DNA strand and the expression will increase.

\section*{73)}
\subsection*{A)} If AraC is a repressor, araC$^-$ mutant cell will have high AraA RNA level.

\subsection*{B)} We see that in the mutant, AraA RNA is always low, to AraC is an activator of AraA.

\section*{74)}
Using only one codon is not very efficient because the translation will be quickly limited by a lack of tRNA.

If we put this gene in E.Coli. the cell will lack of tRNA. To solve this problem, we can simply add the corresponding tRNA.

\section*{75)}
Epigenetic.

\section*{76)}
Intron splicing.

\section*{77)}
The translation will end at the first stop codon and only the first gene will be translated. So, the ribosome have to be able to begin the translation not only at the begin of the mRNA or to continue another protein after stop codon.


\end{document}